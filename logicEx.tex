
\section{Exercises}

\begin{small}
\begin{enumerate}
\newcolumntype{Q}{>{\arraybackslash}m{.45\textwidth}}
\newcolumntype{A}{>{\arraybackslash}m{.5\textwidth}}
%\begin{longtable}{m{0.3\textwidth} || m{0.6\textwidth}}
\begin{longtable}{Q || A}
\hline
\vspace{-.1in}

Let $B$ be the proposition that Joe Biden was elected president in 2020, $C$ be
the proposition that Covid-19 was completely eradicated from the earth in 2021,
and $R$ be the proposition that \textit{Roe v.~Wade} was overturned in 2022.
\item What's $B \vee C$?
&
True.\\
\hline
\vspace{-.15in}
\item What's $B \wedge C$?
&
False.\\
\hline
\vspace{-.15in}
\item What's $B \wedge R$?
&
True.\\
\hline
\vspace{-.15in}
\item What's $B \wedge \neg R$?
&
False.\\
\hline
\vspace{-.15in}
\item What's $\neg C \vee \neg R$?
&
True.\\
\hline
\vspace{-.15in}
\item What's $\neg (C \vee \neg R)$?
&
True.\\
\hline
\vspace{-.15in}
\item What's $\neg (\neg C \vee R)$?
&
False.\\
\hline
\vspace{-.15in}
\item What's $\neg C \vee B$?
&
True.\\
\hline
\vspace{-.15in}
\item What's $\neg C \oplus B$?
&
False.\\
\hline
\vspace{-.15in}
\item What's $\neg C \oplus \neg B$?
&
True.\\
\hline
\vspace{-.15in}
\item What's $\neg \neg \neg \neg B$?
&
True.\\
\hline
\vspace{-.15in}
\item What's $\neg \neg \neg \neg \neg B$?
&
False.\\
\hline
\vspace{-.15in}
\item What's $\neg \neg \neg \neg \neg C$?
&
True.\\
\hline
\vspace{-.15in}
\item What's $B \vee C \vee R$?
&
True.\\
\hline
\vspace{-.15in}
\item What's $B \wedge C \wedge R$?
&
False.\\
\hline
\vspace{-.15in}
\item What's $B \wedge \neg C \wedge R$?
&
True.\\
\hline
\vspace{-.15in}
\item What's $B \Rightarrow R$?
&
True. (Even though there is plainly no causality there.)\\
\hline
\vspace{-.15in}
\item What's $R \Rightarrow B$?
&
True. (Ditto.)\\
\hline
\vspace{-.15in}
\item What's $B \Rightarrow C$?
&
False. (The premise is true, so the conclusion must also be true for this
sentence to be true.)\\
\hline
\vspace{-.15in}
\item What's $C \Rightarrow B$?
&
\textbf{True}. (The premise is false, so all bets are off and the sentence is true.)\\
\hline
\vspace{-.15in}
\item What's $C \Rightarrow \neg R$?
&
\textbf{True}. (The premise is false, so all bets are off and the sentence is true.)\\
\hline
\vspace{-.15in}
\item What's $C \Leftrightarrow B$?
&
False. (The truth values of the left and right sides are not the same.)\\
\hline
\vspace{-.15in}
\item What's $C \Leftrightarrow \neg B$?
&
True. (The truth values of the left and right sides \textit{are} the same.)\\
\hline
\item \label{assertion} I make this assertion:

\begin{center}
``$X \wedge \neg Y \wedge \neg(Z \Rightarrow Q)$.''
\end{center}

And since I'm the professor, you can assume I'm correct about this. From
this information alone, can you determine a unique set of values for the four
variables? Or is there more than one possibility for them?
&
\scriptsize
There is actually only one solution. Here's one way to tell. We know that $X$
must be true, since it's being ``and-ed'' in to another expression. We know
that $Y$ must be false, since its \textit{opposite} is similarly being
``and-ed'' in. Finally, we also know that $Z$ must be true and $Q$ must be
false, since the only way an implication ($\Rightarrow$) can be false is if its
premise is true and its conclusion is false. And the implication here
\textit{must} be false if the professor is telling the truth, because its
\textit{opposite} is being ``and-ed'' in to the three other things. So the one
and only answer is: $X=1, Y=0, Z=1, Q=0$. (You can figure this all out with truth tables too, of course, and for most examples you would. I just wanted to make an exercise that you could figure out in your head without pencil and paper.)
\\
\hline
\item What if I get rid of $Q$ and replace it with $X$, thus making my
assertion:

\begin{center}
``$X \wedge \neg Y \wedge \neg(Z \Rightarrow X)$.''
\end{center}

Now what is/are
the solutions?
&
\scriptsize
Now it's impossible, and if you study the previous item, you'll see why. The
only way that item \ref{assertion} could be true was if the conclusion of the
implication (namely, $Q$) was false. But $X$ had to be true. So whether $X$ is
true or false in this new assertion, something will go haywire: either it'll be
true and the third and-ed thing will be false, or else it'll be false and the
first and-ed thing will be false. There's no way the professor could be telling
the truth here.
\\
\hline
\medskip

At the time of this writing, all professors are human, and that's what
I'll be assuming in these exercises.
\smallskip
\item True or false: $\forall x\ \textsc{Professor}(x)$.
&
False. This says ``everyone and everything is a professor,'' which is clearly
not true. (Consider what you ate for lunch as a counterexample.)\\
\hline
\item True or false: $\forall x\ \textsc{Human}(x)$.
&
False. This says ``everyone and everything is human,'' which is clearly
not true. (Consider the book in front of you as a counterexample.)\\
\hline
\item True or false: $\neg \forall x\ \textsc{Human}(x)$.
&
True. This says ``it's \textit{not} the case that everyone and everything is
human.'' And that certainly is not the case.\\
\hline
\item \label{nothingishuman} True or false: $\forall x\ \neg \textsc{Human}(x)$.
&
False. This says ``nothing is human,'' which is clearly not true. (Consider
yourself as a counterexample.) \\
\hline
\item True or false: $\exists x\ \neg \textsc{Human}(x)$.
&
True. This says ``there's at least one thing in the universe which is not
human.'' (Consider your lunch.)\\
\hline
\item True or false: $\neg \exists x\ \textsc{Human}(x)$.
&
False. This says ``nothing is human,'' just like item~\ref{nothingishuman}
did.\\
\hline
\item True or false: $\forall x\ \textsc{Human}(x) \wedge \textsc{Professor}(x)$.
&
\footnotesize
Not even close. This says ``everything in the universe is a human
professor.'' (Even though I would exist in such a world, what a sad, limited
place it would be.)\\
\hline
\item \footnotesize True or false: $\forall x\ \textsc{Human}(x) \Rightarrow \textsc{Professor}(x)$.
&
\footnotesize
False. This says ``every person is a professor.'' (Consider LeBron James.) Keep
in mind: ``$\forall$'' and ``$\wedge$'' don't really play well together.\\
\hline
\item \footnotesize True or false: $\exists x\ \textsc{Professor}(x) \Rightarrow
\textsc{Human}(x)$.
&
\footnotesize
This is technically true, but for a stupid reason, and whoever wrote it almost
certainly didn't intend what they wrote. It says, ``there's at least one thing
in the universe which either (a) isn't a professor, or (b) if it \textit{is} a
professor, is also human.'' Keep in mind: ``$\exists$'' and ``$\Rightarrow$''
don't really play well together. To drill this lesson home, realize that you
could substitute almost \textit{any} predicates for \textsc{Professor}() and
\textsc{Human}() in that statement and it would still be true. (Try swapping
out \textsc{Professor}() for \textsc{Condiment}() and \textsc{Human}() for
\textsc{AstrologicalSign}(). Now try $x$=EuropeanUnion and voila! the statement
is true. The EU is not a condiment, nor is it an astrological sign, so both
sides of the implication are false, and never forget: false $\Rightarrow$ false
= \textbf{true}.)\\
\hline
\item \label{trueatlast} \footnotesize True or false: $\forall x\ \textsc{Professor}(x) \Rightarrow
\textsc{Human}(x)$.
&
\footnotesize
True at last! This is what we were trying to say all along. Every professor is a
person.\\
\hline
\item \scriptsize True or false: $\neg \exists x\ \textsc{Professor}(x)
\Rightarrow \neg \textsc{Human}(x)$.
&
\footnotesize
True! This is an equivalent statement to item \ref{trueatlast}. There's nothing
in the universe that is a professor yet not a human. (At least, at the time of
this writing.)\\
\hline
\end{longtable}
\end{enumerate}
\end{small}

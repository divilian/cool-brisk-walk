
\chapter{Proof}

We've seen a lot of pretty sights on our cool brisk walk. We've caught a
glimpse of the simple elegance of sets and relations, the precision of
probabilistic reasoning, the recursive structure of trees, the explosive
nature of combinatorics, and much more. None of these things have we
plumbed to the depths, but we've appreciated their beauty and taken note of
where they stood along our blazed trail. You'll remember this hike when you
run into such concepts again and again in future computer science and
math courses, and in your career beyond academics.

Now we have one more stop to make before returning to the trailhead, and
that deals with the notion of \textit{proof}. As we've studied these
various mathematical entities, I've pointed out certain of their
properties. A free tree has one more vertex than edge, for example. The
cardinality of the union of two sets is at least as big as each of their
individual unions. If you flip-all-the-bits-and-add-one in a two's
complement scheme, and then perform that flip-and-add operation again,
you'll return to the original number. But with a few exceptions, we haven't
\textit{proven} any of these things. I've just stated them, and you've
taken them on faith.

\index{proof}
In order to establish reliable truth, of course, professional
mathematicians aren't satisfied with unsubstantiated statements. They need
to be convinced that the claims we make do truly hold, and provably so, in
all circumstances. What they seek is a \textbf{proof} of a claim: an
irrefutable sequence of logical steps that leads inescapably from our
premises to our conclusion. There are several ways to construct a
convincing proof, and this chapter will highlight some of them.

Most authors of discrete math texts, by the way, interweave the concept of
proof throughout the entire book. I'm taking a radical departure by
deferring this fundamental idea until the very end. Why did I make this
choice? A couple of reasons. First, as I said at the very beginning, my
target audience for this book is future practitioners, not theoretical
researchers. I think most practicing computer scientists need fluency with
the tools of discrete math, not the ability to devise new fundamental
theorems about them. We mostly need to use, not to prove. The second reason
is that I've found that interspersing proofs throughout the presentation
often distracts the reader from the concepts at hand, since the focus
shifts slightly from the concept being discussed (the function, the
directed graph, what have you) to the proof \textit{about} the concept.
When the proof itself takes center stage, it forces the actual subject
matter to share the limelight. And with technical material like this, we
need all the light we can get.

\section{Proof concepts}

A proof is essentially a chain of reasoning, in which each step can be
logically deduced from the ones that preceded it. It's a way of putting
your thought process on display so it can be scrutinized to make sure it
holds water. Any step of your reasoning which was unwarranted will be
exposed, and perhaps reveal that the conclusion you thought was true isn't
necessarily dependable after all.

Here's an example from everyday life. I'm driving home from work one
afternoon, and I believe that my wife and children will be gone when I
arrive. I'll be coming home to an empty house. 

Now why do I believe this? Well, if I unravel my reasoning, it goes like
this. First, today is Wednesday. On Wednesday nights, my wife and children
normally go to church for dinner and service. Second, my wife likes to call
me ahead of time if this plan changes. My cell phone is in my pocket, and
has not rung, and so I conclude that the plan has not changed. I look at my
watch, and it reads 5:17pm, which is after the time they normally leave, so
I know I'm not going to catch them walking out the door. This is, roughly
speaking, my thought process that justifies the conclusion that the house
will be empty when I pull into the garage.

Notice, however, that this prediction depends precariously on several
facts. What if I spaced out the day of the week, and this is actually
Thursday? All bets are off. What if my cell phone battery has run out of
charge? Then perhaps she \textit{did} try to call me but couldn't reach me.
What if I set my watch wrong and it's actually 4:17pm? \textit{Etc.} Just
like a chain is only as strong as its weakest link, a whole proof falls
apart if even one step isn't reliable.

\index{artificial intelligence (AI)}
Knowledge bases in artificial intelligence systems are designed to support
these chains of reasoning. They contain statements expressed in formal
logic that can be examined to deduce \textit{only} the new facts that
logically follow from the old. Suppose, for instance, that we had a
knowledge base that currently contained the following facts:
\begin{center}
\begin{enumerate}
\item A$\Rightarrow$C
\item $\neg$(C$\wedge $D)
\item (F$\vee \neg$E)$\Rightarrow$D
\item A$\vee$B
\end{enumerate}
\end{center}
\index{propositional logic}
These facts are stated in propositional logic, and we have no idea what any
of the propositions really \textit{mean}, but then neither does the
computer, so hey. Fact \#1 tells us that if proposition A (whatever that
may mean) is true, then we know C is true as well. Fact \#2 tells us that
we know C$\wedge$D is false, which means at least one of the two must be
false. And so on. Large knowledge bases can contain thousands or even
millions of such expressions.  It's a complete record of everything the
system ``knows."

Now suppose we learn an additional fact: $\neg$B. In other words, the
system interacts with its environment and comes to the conclusion that
proposition B must be false. What else, if anything, can now be safely
concluded from this?

It turns out that we can now conclude that \textit{F is also false.} How
do we know this? Here's how:
\begin{enumerate}
\index{disjunctive syllogism}
\item Fact \#4 says that either A or B (or both) is true. But we just
discovered that B was false. So if it ain't B, it must be A, and therefore
we conclude that \textbf{A must be true}. (For the curious, this rule of
common sense is called a ``disjunctive syllogism.")
\index{\textit{modus ponens}}
\item Now if A is true, we know that C must also be true, because fact \#1
says that A implies C. So we conclude that \textbf{C is true.} (This one
goes by the Latin phrase ``\textit{modus ponens}.")
\index{\textit{modus tollens}}
\index{disjunction}
\item Fact \#2 says that C$\wedge$D must be \textit{false}. But we just
found out that C was true, so it must be D that's false in order to make
the conjunction false. So we conclude that \textbf{D is false.} (This is a
disjunctive syllogism in disguise, combined with De Morgan's law.)
\item Finally, fact \#3 tells us that if either F were true or E were false,
then that would imply that D would be true. But we just found out that D is
false. Therefore, neither F nor $\neg$E can be true. (This step combines
``\textit{modus tollens}" with ``disjunction elimination.")
So we conclude that \textbf{F must be false}. \textit{Q.E.D.}
\end{enumerate}
\index{\textit{quod erat demonstrandum (Q.E.D.)}}
(The letters ``\textit{Q.E.D.}" at the end of a proof stand for a Latin
phrase meaning, ``we just proved what we set out to prove." It's kind of a
way to flex your muscles as you announce that you're done.)

Not all proofs are performed in formal logic like this; some use algebra,
set theory, or just plain English. But the idea is the same: start with
what you know, proceed to derive new knowledge using only legal operations,
and end with your conclusion.

\index{axioms}
\index{postulates}
The things we're allowed to start with are called \textbf{axioms} (or
\textbf{postulates}). An axiom is a presupposition or definition that is
\textit{given} to be true, and so it is legal grounds from which to start.
A proof can't even get off the ground without axioms. For instance, in
step~1 of the above proof, we noted that either A or B must be true, and so
if B isn't true, then A must be.  But we couldn't have taken this step
without knowing that disjunctive syllogism is a valid form of reasoning.
It's not important to know all the technical names of the rules that I
included in parentheses. But it is important to see that we made use of an
axiom of reasoning on every step, and that if any of those axioms were
incorrect, it could lead to a faulty conclusion.

\index{theorems}
\index{distributive}
\index{Venn diagrams}
\index{union (of sets)}
\index{intersection (of sets)}
When you create a valid proof, the result is a new bit of knowledge called
a \textbf{theorem} which can be used in future proofs. Think of a theorem
like a subroutine in programming: a separate bit of code that does a job and
can be invoked at will in the course of doing other things. One theorem we
learned in chapter~\ref{chap:sets} was the distributive property of sets;
that is, that X $\cap$ (Y $\cup$ Z) = (X $\cap$ Y) $\cup$ (X $\cap$ Z).
This can be proven through the use of Venn diagrams, but once you've proven
it, it's accepted to be true, and can be used as a ``given" in future
proofs.

\section{Types of proof}

There are a number of accepted ``styles" of doing proofs. Here are some
important ones:

\subsection{Direct proof}
\index{direct proof}

The examples we've used up to now have been \textbf{direct proof}s. This is
where you start from what's known and proceed directly by positive steps
towards your conclusion.

\index{word ladders}
\index{Carroll, Lewis}
Direct proofs remind me of a game called ``word ladders," invented by Lewis
Carroll, that you might have played as a child:

\begin{center}
\texttt{WARM} \\
\texttt{||||} \\
\texttt{????} \\
\texttt{||||} \\
\texttt{COLD}
\end{center}
You start with one word (like \texttt{WARM}) and you have to come up with a
sequence of words, \textit{each of which differs from the previous by only
one letter}, such that you eventually reach the ending word (like
\texttt{COLD}). It's sort of like feeling around in the dark:
\begin{center}
\texttt{WARM} \\
\texttt{WAR\underline{T}} \\
\texttt{WA\underline{L}T} \\
\texttt{W\underline{I}LT} \\
\texttt{WIL\underline{D}} \\
\texttt{||||} \\
\texttt{....}
\end{center}
This attempt seemed promising at first, but now it looks like it's going
nowhere.  (``\texttt{WOLD}?" ``\texttt{CILD}?" Hmm....) After starting over
and playing around with it for a while, you might stumble upon:
\begin{center}
\texttt{WARM} \\
\texttt{W\underline{O}RM} \\
\texttt{WOR\underline{D}} \\
\texttt{\underline{C}ORD} \\
\texttt{CO\underline{L}D}
\end{center}
This turned out to be a pretty direct path: for each step, the letter we
changed was exactly what we needed it to be for the target word
\texttt{COLD}. Sometimes, though, you have to meander away from the target
a little bit to find a solution, like going from \texttt{BLACK} to
\texttt{WHITE}:
\begin{nobreak}
\begin{center}
\texttt{BLACK} \\
\texttt{\underline{C}LACK} \\
\texttt{C\underline{R}ACK} \\
\texttt{\underline{T}RACK} \\
\texttt{TR\underline{I}CK} \\
\texttt{TRIC\underline{E}} \\
\texttt{TRI\underline{T}E} \\
\texttt{\underline{W}RITE} \\
\texttt{W\underline{H}ITE}
\end{center}
\end{nobreak}
Here, we had to temporarily change our first letter three different times
--- two of which seemingly brought us no nearer to \texttt{WHITE} --- in
order to successfully forge a path through the tangled forest.

\index{axioms}
Knowing which direction to set out on is a matter of intuition plus trial
and error. Given the axioms of any system (whether algebra, predicate
logic, sets, \textit{etc.}) there are an unfathomable number of different
ways to proceed. The vast majority of them are bound to lead to dead ends.
This is why a valid proof, when it is finished, is often an elegant and
beautiful thing. It's a thin braid of jewels glistening in the midst of a
whole lot of mud.

\subsection{Indirect proof}
\index{indirect proof}
\index{proof by contradiction}
\index{\textit{reductio ad absurdum}}

Also known as a \textbf{proof by contradiction} or \textbf{\textit{reductio
ad absurdum}}, the \textbf{indirect proof} starts in a completely opposite
way. It says, ``okay, I'm trying to prove X. Well, suppose for the sake of
argument I assume that the opposite --- \textit{not} X --- is true. Where
would that lead me?" If you follow all the rules and it leads you to a
contradiction, this tells you that the original assumption of $\neg$X must
have been false. And this in turn proves that X must be true.

We do this all the time in our thinking. Say you're driving down the
highway. How do you \textit{know} that the alternator in your car engine is
working? A direct proof would require that you open the hood and examine
the part, testing to ensure it works properly. An indirect proof simply
says, ``well, suppose it \textit{weren't} working properly. Then, my car
engine wouldn't operate. But here I am, driving down the road, and the
engine obviously \textit{does} operate, so that tells me that the
alternator must be working properly."

\index{Euclid}
One of the most famous indirect proofs dates from Euclid's
\textit{Elements} in 300 B.C. It proves that the square root of 2 is an
irrational number, a great surprise to mathematicians at the time (most of
whom doubted the very existence of irrational numbers). Remember that an
irrational number is one that \textit{cannot} be expressed as the ratio of
two integers, no matter what the integers are.

Proving this directly seems pretty hard, since how do you prove that there
\textit{aren't} any two integers whose ratio is $\sqrt{2}$, no matter how
hard you looked? I mean, 534,927 and 378,250 are pretty dang close:
\[
\Bigg(\frac{534,927}{378,250}\Bigg)^2 = 2.000005.
\]
How could we possibly prove that no matter how hard we look, we can never
find a pair that will give it to us exactly?

One way is to assume that $\sqrt{2}$ \textit{is} a rational number, and
then prove that down that path lies madness. It goes like this. Suppose
$\sqrt{2}$ is rational, after all. That means that there must be two
integers, call them $a$ and $b$, whose ratio is exactly equal to
$\sqrt{2}$:
\[
\frac{a}{b} = \sqrt{2}.
\]
This, then, is the starting point for our indirect proof. We're going to
proceed under this assumption and see where it leads us.

By the way, it's clear that we could always reduce this fraction to lowest
terms in case it's not already. For instance, if $a=6$ and $b=4$, then our
fraction would be $\frac{6}{4}$, which is the same as $\frac{3}{2}$, so we
could just say $a=3$ and $b=2$ and start over. Bottom line: if $\sqrt{2}$
is rational, then we can find two integers $a$ and $b$ that have no common
factor (if they do have a common factor, we'll just divide it out of both
of them and go with the new numbers) whose ratio is $\sqrt{2}$.

Okay then. But now look what happens. Suppose we square both sides of the
equation (a perfectly legal thing to do):
\begin{align*}
\frac{a}{b} &= \sqrt{2} \\
\Bigg(\frac{a}{b}\Bigg)^2 &= (\sqrt{2})^2 \\
\frac{a^2}{b^2} &= 2 \\
a^2 &= 2{b^2}. \\
\end{align*}
Now if $a^2$ equals 2 times something, then $a^2$ is an even number. But
$a^2$ can't be even unless $a$ itself is even. (Think hard about that one.)
This proves, then, that $a$ is even. Very well. It must be equal to twice
some other integer. Let's call that $c$. We know that $a=2c$, where $c$ is
another integer. Substitute that into the last equation and we get:
\begin{align*}
(2c)^2 &= 2{b^2} \\
4c^2 &= 2{b^2} \\
2c^2 &= b^2. \\
\end{align*}
So it looks like $b^2$ must be an even number as well (since it's equal to
2 times something), and therefore $b$ is also even. But wait a minute. We
started by saying that $a$ and $b$ \textit{had no common factor}. And now
we've determined that they're both even numbers! This means they both have
a factor of 2, which contradicts what we started with. The only thing we
introduced that was questionable was the notion that there \textit{are} two
integers $a$ and $b$ whose ratio was equal to $\sqrt{2}$ to begin with.
That must be the part that's faulty then. Therefore, $\sqrt{2}$ is
\textit{not} an irrational number. \textit{Q.E.D.}

% DUH how about a DISCRETE MATH example???

\section{Proof by induction}
\index{mathematical induction}

\index{recursion}
\index{rooted trees}
One of the most powerful methods of proof --- and one of the most difficult
to wrap your head around --- is called \textbf{mathematical induction}, or
just ``induction" for short. I like to call it ``proof by recursion,"
because this is exactly what it is. Remember that we discussed recursion in
the context of rooted trees (see p.\pageref{recursion}). A tree can be thought
of as a node with several children --- each of which are, in turn, trees.
Each of \textit{them} is the root node of a tree comprised of yet
smaller trees, and so on and so forth. If you flip back to the left-hand
side of Figure~\ref{rootedtree} on p.\pageref{page:rootedtree}, you'll see that
A is the root of one tree, and its two children, F and B, are roots of
their own smaller trees in turn. If we were to traverse this tree in (say)
pre-order, we'd visit the root, then visit the left and right subtrees in
turn, treating each of them as their \textit{own} tree. In this way we've
broken up a larger problem (traversing the big tree) into smaller problems
(traversing the smaller trees F and B). The A node has very little to do:
it just visits itself, then defers all the rest of the work onto its
children. This idea of pawning off most of the work onto smaller
subproblems \textit{that you trust will work} is key to the idea of
inductive proofs.

Mathematical induction is hard to wrap your head around because it feels
like cheating. It seems like you never actually prove anything: you defer
all the work to someone else, and then declare victory. But the chain of
reasoning, though delicate, is strong as iron.

\subsection{Casting the problem in the right form}

\index{predicates}
\index{natural numbers ($\mathbb{N}$)}
Let's examine that chain. The first thing you have to be able to do is
express the thing you're trying to prove as \textit{a predicate about
natural numbers}. In other words, you need to form a predicate that has one
input, which is a natural number. You're setting yourself up to prove that
the predicate is true \textit{for all natural numbers.} (Or at least, all
natural numbers of at least a certain size.)

\index{drinking age}
\index{voting age}
Suppose I want to prove that in the state of Virginia, all legal drinkers
can vote. Then I could say ``let \textsc{Vote}($n$) be the proposition that
a citizen of age $n$ can vote."

If I want to prove an algebraic identity, like $\sum\limits_{i=1}^x{i} =
\frac{x(x+1)}{2}$, then I have to figure out which variable is the one that
needs to vary across the natural numbers. In this case it's the $x$
variable in my equation. So I'll say ``let P($n$) be the proposition that
$\sum\limits_{i=1}^n{i} = \frac{n(n+1)}{2}$." (The choice of the letter ``$n$"
isn't important here --- it just needs to be a letter that stands for a
number. We could have chosen anything, even sticking with $x$. Later, we'll
use ``$k$" as a stand-in, so keep your eyes peeled for that.)

If I want to prove that the number of leaves in a perfect binary tree is
one more than the number of internal nodes, I'd have to think about which
\textit{quantity} I can parameterize on (\textit{i.e.}, which quantity I
can use for my $n$.) In this case, I'd probably use the \textit{height} of
the tree. I'd say ``let P($n$) be the proposition that the number of leaves
in a perfect binary tree of height $n$ is one more than the number of
internal nodes."

These are just examples. In any case, you need to cast your proof in a form
that allows you to make statements in terms of the natural numbers. Then
you're ready to begin the process of proving by induction that your
predicate is true for \textit{all} the natural numbers.

\subsection{Proof by induction: weak form}
\index{weak form of induction}

There are actually two forms of induction, the weak form and the strong
form. Let's look at the \textbf{weak form} first. It says:
\begin{enumerate}
\item \label{basecaseweak} \textit{If} a predicate is true for a certain number,
\item \label{inductivestepweak} \textit{and} its being true for some number
would reliably mean that it's also true for the next number (\textit{i.e.},
one number greater), 
\item \textit{then} it's true for all numbers.
\end{enumerate}
All you have to do is prove those two things, and you've effectively proven
it for every case.

\index{base case (of a proof)}
\index{inductive step}
The first step is called the \textbf{base case}, and the ``certain number"
we pick is normally either 0 or 1. The second step, called the
\textbf{inductive step}, is where all the trouble lies. You have to look
really, really carefully at how it's worded, above. We are \textit{not}
assuming that the predicate is true for any old number! We are simply
considering, \textit{if} it's true for any old number, whether that would
necessarily imply it's also true for the next number. In terms of the
predicate, we're asking ``does P($k$) imply P($k+1$)?" In other words: ``we
aren't sure if P($k$) is true. But if it is --- a big ``if," of course
--- would that logically demand that P($k+1$) was also true?" If you can
prove that it does, then you're in business.

\index{dominos}
The whole thing is set up like a row of dominos. If one domino falls, then
the one after it will also fall. And if that one falls, then so will the
next. All that is needed is a base case to tip over the first domino, and
by this trail of causality, \textit{all} the dominos will fall.

\index{inductive hypothesis}
One terminology note: the entire second step is called the inductive step,
but the first half of it (the part where we assume that P($k$) is true) is
called the \textbf{inductive hypothesis}. We never prove the inductive
hypothesis; rather, we assume it, and then see if that allows us to deduce
that P($k+1$) would also be true.


\subsubsection{Example 1}

\index{drinking age}
\index{voting age}
Let's work this out for the drinking/voting example. Let \textsc{Vote}($n$)
be the proposition that a citizen of age $n$ can vote. Our proof goes like
this:
\begin{enumerate} 

\item \textbf{base case.} \textsc{Vote}(21) is true, because a 21-year old
is old enough to vote in the state and national elections.

\item \textbf{inductive step.}
\textsc{Vote}(k)$\Rightarrow$\textsc{Vote}(k+1). Why? Because nobody's
gettin' any younger. If you can vote in a particular year, then you're also
old enough to vote next year. Unless the laws change, there will never be a
case when someone old enough to vote this year turns out to be too young to
vote next year.

\item \textbf{conclusion.} Wow. $\forall n\geq21\ \textsc{Vote}(n)$. We're done. \textit{Q.E.D.} and all that.
\end{enumerate}
The only specific example we showed was true was \textsc{Vote}(21). And yet
we managed to prove \textsc{Vote}($n$) for \textit{any} number $n\geq21$.

\index{inductive step}
\index{inductive hypothesis}
Let's look back at that inductive step, because that's where all the action
is. It's crucial to understand what that step does \textit{not} say. It
doesn't say ``\textsc{Vote}($k$) is true for some number $k$." If it did,
then since $k$'s value is arbitrary at that point, we would basically be
assuming the very thing we were supposed to prove, which is circular
reasoning and extremely unconvincing. But that's not what we did. Instead,
we made the inductive hypothesis and said, ``okay then, let's assume for a
second a 40-year-old can vote. We don't know for sure, but let's say she
can. Now, if that's indeed true, can a 41-year-old also vote? The answer is
yes." We might have said, ``okay then, let's assume for a second a
7-year-old can vote. We don't know for sure, but let's say she can. Now, if
that's indeed true, can an 8-year-old also vote? The answer is yes." Note
carefully that we did \textit{not} say that 8-year-olds can vote! We merely
said that \textit{if} 7-year-olds can, why then 8-year-olds must be able to
as well. Remember that X$\Rightarrow$Y is true if either X is false or Y is
true (or both). In the 7/8-year-old example, the premise X turns out to be
false, so this doesn't rule out our implication.

The result is a row of falling dominos, up to whatever number we wish. Say
we want to verify that a \textbf{25-year-old} can vote. Can we be sure? Well:
\begin{enumerate}
\item If a 24-year-old can vote, then that would sure prove it (by the
inductive step).
\item So now we need to verify that a 24-year-old can vote. Can he? Well,
if a 23-year-old can vote, then that would sure prove it (by the inductive
step).
\item Now everything hinges on whether a 23-year-old can vote. Can he? Well,
if a 22-year-old can vote, then that would sure prove it (by the inductive
step).
\item So it comes down to whether a 22-year-old can vote. Can he? Well,
if a 21-year-old can vote, then that would sure prove it (by the inductive
step).
\item And now we need to verify whether a 21-year-old can vote. Can he? Yes
(by the base case).
\end{enumerate}

\subsubsection{Example 2}

\index{Gauss, Carl Friedrich}
A famous story tells of Carl Friedrich Gauss, perhaps the most brilliant
mathematician of all time, getting in trouble one day as a schoolboy. As
punishment, he was sentenced to tedious work: adding together all the
numbers from 1 to 100. To his teacher's astonishment, he came up with the
correct answer in a moment, not because he was quick at adding integers,
but because he recognized a trick. The first number on the list (1) and the
last (100) add up to 101. So do the second number (2) and the
second-to-last (99). So do 3 and 98, and so do 4 and 97, \textit{etc.}, all
the way up to 50 and 51. So really what you have here is 50 different sums
of 101 each, so the answer is $50\times 101 = 5050$. In general, if you add
the numbers from 1 to $x$, where $x$ is any integer at all, you'll get
$\frac{x}{2}$ sums of $x+1$ each, so the answer will be $\frac{x(x+1)}{2}$.

Now, use mathematical induction to prove that Gauss was right
(\textit{i.e.}, that $\sum\limits_{i=1}^x{i} = \frac{x(x+1)}{2}$) for all numbers
$x$.

First we have to cast our problem as a predicate about natural numbers.
This is easy: we say ``let P($n$) be the proposition that $\sum\limits_{i=1}^n{i}
= \frac{n(n+1)}{2}$."

Then, we satisfy the requirements of induction:

\begin{enumerate} 

\item \textbf{base case.} We prove that P(1) is true simply by plugging it
in. Setting $n=1$ we have
\begin{align*}
\sum_{i=1}^1{i} & \stackrel{?}{=} \frac{1(1+1)}{2} \\
1 & \stackrel{?}{=} \frac{1(2)}{2} \\
1 &= 1 \quad \checkmark
\end{align*}

\item \textbf{inductive step.}
We now must prove that P($k$)$\Rightarrow$P($k+1$). Put another way, we
\textit{assume} P($k$) is true, and then use that assumption to prove that
P($k+1$) is also true.

Let's be crystal clear where we're going with this. Assuming that P($k$) is
true means we can count on the fact that
\begin{align*}
1+2+3+\cdots+k = \frac{k(k+1)}{2}.
\end{align*}

What we need to do, then, is prove that P($k+1$) is true, which amounts to
proving that
\begin{align*}
1+2+3+\cdots+(k+1) = \frac{(k+1)((k+1)+1)}{2}.
\end{align*}

Very well. First we make the inductive hypothesis, which allows us to
assume:
\begin{align*}
1+2+3+\cdots+k = \frac{k(k+1)}{2}.
\end{align*}
The rest is just algebra. We add $k+1$ to both sides of the equation, then
multiply things out and factor it all together. Watch carefully:
\begin{align*}
1+2+3+\cdots+k+(k+1) &= \frac{k(k+1)}{2} + (k+1) \\
&= \frac{1}{2}k^2 + \frac{1}{2}k + k + 1 \\
&= \frac{1}{2}k^2 + \frac{3}{2}k + 1 \\
&= \frac{k^2+3k+2}{2} \\
&= \frac{(k+1)(k+2)}{2} \\
&= \frac{(k+1)((k+1)+1)}{2}. \quad \checkmark
\end{align*}
\item \textbf{conclusion.} Therefore, $\forall n\geq1 \ \text{P}(n)$.
\end{enumerate}


\subsubsection{Example 3}

Another algebra one. You learned in middle school that $(ab)^n=a^n b^n$.
Prove this by mathematical induction.

Solution: Let P($n$) be the proposition that $(ab)^n=a^n b^n$. 

\begin{enumerate}
\item \textbf{base case.} We prove that P(1) is true simply by plugging it
in. Setting $n=1$ we have
\begin{align*}
(ab)^1 & \stackrel{?}{=} a^1 b^1 \\
ab &= ab \quad \checkmark
\end{align*}

\item \textbf{inductive step.}
We now must prove that P($k$)$\Rightarrow$P($k+1$). Put another way, we
\textit{assume} P($k$) is true, and then use that assumption to prove that
P($k+1$) is also true.

Let's be crystal clear where we're going with this. Assuming that P($k$) is
true means we can count on the fact that
\begin{align*}
(ab)^k = a^k b^k.
\end{align*}

What we need to do, then, is prove that P($k+1$) is true, which amounts to
proving that
\begin{align*}
(ab)^{k+1} = a^{k+1} b^{k+1}.
\end{align*}

Now we know by the very definition of exponents that:
\begin{align*}
(ab)^{k+1} &= ab(ab)^k. \\
\end{align*}

Adding in our inductive hypothesis then lets us determine:
\begin{align*}
(ab)^{k+1} &= ab(ab)^k \\
&= ab \cdot a^k b^k \\
&= a \cdot a^k \cdot b \cdot b^k \\
&= a^{k+1} b^{k+1} \quad \checkmark
\end{align*}
\item \textbf{conclusion.} Therefore, $\forall n\geq1 \ \text{P}(n)$.
\end{enumerate}


\subsubsection{Example 4}

\index{perfect binary tree}
\index{leaves}
\index{internal nodes}
Let's switch gears and talk about structures. Prove that the number of
leaves in a perfect binary tree is one more than the number of internal
nodes.

Solution: let P($n$) be the proposition that a perfect binary tree of
height $n$ has one more leaf than internal node. That is, if $l_k$ is the
number of \textit{l}eaves in a tree of height $k$, and $i_k$ is the number of
\textit{i}nternal nodes in a tree of height $k$, let P($n$) be the
proposition that $l_n = i_n + 1$.

\begin{enumerate}
\item \textbf{base case.} We prove that P(0) is true simply by inspection.
If we have a tree of height 0, then it has only one node (the root). This
sole node is a leaf, and is not an internal node. So this tree has 1 leaf,
and 0 internal nodes, and so $l_0 = i_0 + 1$. \quad \checkmark

\item \textbf{inductive step.}
We now must prove that P($k$)$\Rightarrow$P($k+1$). Put another way, we
\textit{assume} P($k$) is true, and then use that assumption to prove that
P($k+1$) is also true.

Let's be crystal clear where we're going with this. Assuming that P($k$) is
true means we can count on the fact that
\begin{align*}
l_k = i_k + 1.
\end{align*}

What we need to do, then, is prove that P($k+1$) is true, which amounts to
proving that
\begin{align*}
l_{k+1} = i_{k+1} + 1.
\end{align*}

We begin by noting that the number of nodes \textit{on level $k$} of a perfect
binary tree is $2^k$. This is because the root is only one node, it has two
children (giving 2 nodes on level 1), both those children have two children
(giving 4 nodes on level 2), all four of those children have two children
(giving 8 nodes on level 3), \textit{etc.} Therefore, $l_k = 2^k$, and
$l_{k+1} = 2^{k+1}$.

Further, we observe that $i_{k+1} = i_k + l_k$: this is just how trees
work. In words, suppose we have a perfect binary tree of height $k$, and we
add another level of nodes to it, making it a perfect binary tree of height
$k+1$. Then \textit{all} of the first tree's nodes (whether internal or
leaves) become internal nodes of bigger tree.

Combining these two facts, we have $i_{k+1} = i_k + 2^k$. By the inductive
hypothesis, we assume that $2^k = i_k + 1$, and we now must prove that
$2^{k+1} = i_{k+1} + 1$. Here goes:
\begin{align*}
i_{k+1} &= i_k + 2^k &\textsl{(property of trees)} \\
i_{k+1} &= 2^k - 1 + 2^k &\textsl{(using inductive hypothesis)} \\
i_{k+1} + 1 &= 2^k + 2^k \\
i_{k+1} + 1 &= 2(2^k) \\
i_{k+1} + 1 &= 2^{k+1}. \quad \checkmark
\end{align*}
\item \textbf{conclusion.} Therefore, $\forall n\geq0 \ \text{P}(n)$.
\end{enumerate}

\subsection{Proof by induction: strong form}
\index{strong form of induction}

Now sometimes we actually need to make a stronger assumption than just
``the single proposition P($k$) is true" in order to prove that P($k+1$) is
true. In all the examples above, the $k+1$ case flowed directly from the
$k$ case, and only the $k$ case. But sometimes, you need to know that
\textit{all} the cases less than $k+1$ are true in order to prove the $k+1$
case. In those situations, we use the \textbf{strong form} of mathematical
induction. It says:

\index{base case (of a proof)}
\index{inductive step}
\begin{enumerate}
\item \label{basecasestrong} \textit{If} a predicate is true for a certain number,
\item \label{inductivestepstrong} \textit{and} its being true for \textit{all
numbers up to and including some number} would reliably mean that it's also
true for the next number (\textit{i.e.}, one number greater), 
\item \textit{then} it's true for all numbers.
\end{enumerate}
\index{inductive hypothesis}
It's exactly the same as the weak form, except that the inductive
hypothesis is stronger. Instead of having to prove
\begin{center}
P($k$)$\Rightarrow$P($k+1$),
\end{center}
we get to prove
\begin{center}
($\forall i\leq k$ P($i$))$\Rightarrow$P($k+1$).
\end{center}
At first glance that might not seem any easier. But if you look carefully,
you can see that we've \textit{added information} to the left hand side of
the implication.  No longer do we need to rely on the single fact that P(5)
is true in order to prove P(6). Now we get to take advantage of the fact
that P(1), P(2), P(3), P(4), and P(5) are \textit{all} known to be true
when we try to prove P(6). And that can make a world of difference.


\subsubsection{Example 1}

The Fundamental Theorem of Arithmetic says that every natural number
(greater than 2) is expressible as the product of one or more primes.  For
instance, 6 can be written as ``$2 \cdot 3$", where 2 and 3 are primes. The
number 7 is itself prime, and so can be written as ``$7$." The number 9,180
can be written as ``$2 \cdot 2 \cdot 3 \cdot 3 \cdot 3 \cdot 5 \cdot 17$," all of
which are primes. How can we prove that this is always possible, no matter
what the number?

Let P($n$) be the proposition that the number $n$ can be expressed as a
product of prime numbers. Our proof goes like this:

\begin{enumerate} 

\item \textbf{base case.} P(2) is true, since 2 can be written as ``2," and
2 is a prime number. (Note we didn't use 0 or 1 as our base case here,
since actually neither of those numbers is expressible as a product of
primes. Fun fact.)

\item \textbf{inductive step.}
We now must prove that ($\forall i \leq k$ \ P($i$))$\Rightarrow$P($k+1$).
Put another way, we \textit{assume} that P($i$) is true for every number up
to $k$, and then use that assumption to prove that P($k+1$) is true as
well.

Regarding the number $k+1$, there are two possibilities: either it's prime,
or it's not. If it is, then we're done, because it can obviously be written
as just itself, which is the product of one prime. (23 can be written as
``23.") But suppose it's not.
Then, it can be broken down as the product of two numbers, each less than
itself. (21 can be broken down as $7 \cdot 3$; 24 can be broken down as $6
\cdot 4$ or $12 \cdot 2$ or $8 \cdot 3$, take your pick.) Now we know nothing
special about those two numbers\dots \textit{except} the fact that the
inductive hypothesis tells us that \textit{all} numbers less than $k+1$ are
expressible as the product of one or more primes! So these two numbers,
whatever they may be, are expressible as the product of primes, and so when
you multiply them together to get $k+1$, you will have a longer string of
primes multiplied together. Therefore, ($\forall i \leq k$ \
P($k$))$\Rightarrow$P($k+1$).

\item \textbf{conclusion.}Therefore, by the strong form of mathematical induction, $\forall n \geq 2$
  \ P($n$).
\end{enumerate}

You can see why we needed the strong form here. If we wanted to prove that
15 is expressible as the product of primes, knowing that 14 is expressible
as the product of primes doesn't do us a lick of good. What we needed to
know was that 5 and 3 were expressible in that way. In general, the strong
form of induction is useful when you have to break something into smaller
parts, but there's no guarantee that the parts will be ``one less" than the
original. You only know that they'll be \textit{smaller} than the original.
A similar example follows.

\subsubsection{Example 2}

\index{free trees}
\index{edges}
\index{nodes (of a graph)}
Earlier (p.\pageref{onelessedge}) we stated that every free tree has one less
edge than node. Prove it.

Let P($n$) be the proposition that a free tree with $n$ nodes has $n-1$
edges.

\begin{enumerate} 

\item \textbf{base case.} P(1) is true, since a free tree with 1 node is
just a single lonely node, and has no edges.

\item \textbf{inductive step.}
We now must prove that ($\forall i \leq k$ \ P($i$))$\Rightarrow$P($k+1$).
Put another way, we assume that all trees \textit{smaller} than the one
we're looking at have one more node than edge, and then use that
assumption to prove that the tree we're looking at also has one more node
than edge. 

We proceed as follows. Take any free tree with $k+1$ nodes. Removing any
edge gives you \textit{two} free trees, each with $k$ nodes or less. (Why?
Well, if you remove any edge from a free tree, the nodes will no longer be
connected, since a free tree is ``minimally connected" as it is. And we
can't break it into \textit{more} than two trees by removing a single edge,
since the edge connects exactly two nodes and each group of nodes on the
other side of the removed edge are still connected to each other.)

Now the sum of the nodes in these two smaller trees is still $k+1$. (This
is because we haven't removed any nodes from the original free tree ---
we've simply removed an edge.) If we let $k_1$ be the number of nodes in
the first tree, and $k_2$ the number of nodes in the second, we have $k_1 +
k_2 = k + 1$.

Okay, but how many \textit{edges} does the first tree have? Answer: $k_1 -
1$. How do we know that? \textit{By the inductive hypothesis.} We're
assuming that any tree smaller than $k+1$ nodes has one less edge than
node, and so we're taking advantage of that (legal) assumption here.
Similarly, the second tree has $k_2 - 1$ edges.

The total number of edges in these two trees is thus $k_1 - 1 + k_2 - 1$,
or $k_1 + k_2 - 2$. Remember that $k+1 = k_1 + k_2$ (no nodes removed), and
so this is a total of $k+1-2 = k-1$ edges.

Bingo. \textit{Removing} one edge from our original tree of $k+1$ nodes
gave us a total of $k-1$ edges. Therefore, that original tree must have had
$k$ edges. We have now proven that a tree of $k+1$ nodes has $k$ edges,
assuming that all smaller trees also have one less edge than node.

\item \textbf{conclusion.} Therefore, by the strong form of mathematical induction, $\forall n \geq 1$
\ P($n$).

\end{enumerate}
\section{Final word}

Finding proofs is an art. In some ways, it's like programming: you have a
set of building blocks, each one defined very precisely, and your goal is
to figure out how to assemble those blocks into a structure that starts
with only axioms and ends with your conclusion. It takes skill, patience,
practice, and sometimes a little bit of luck. 

\index{Appel, Kenneth}
\index{Haken, Wolfgang}
\index{Wiles, Andrew}
\index{Goldbach, Christian}
\index{P=NP?}
Many mathematicians spend years pursuing one doggedly difficult proof, like
Appel and Haken who finally cracked the infamous four-color map problem in
1976, or Andrew Wiles who solved Fermat's Last Theorem in 1994. Some famous
mathematical properties may never have proofs, such as Christian Goldbach's
1742 conjecture that every even integer is the sum of two primes, or the most
elusive and important question in computing theory: does P=NP? (Put very
simply: if you consider the class of problems where it's easy to verify a
solution once you have it, but crazy hard to find it in the first place, is
there actually an easy algorithm for finding the solution that we just haven't
figured out yet?) Most computer scientists think ``no," but despite a
mind-boggling number of hours invested by the brightest minds in the world, no
one has ever been able to prove it one way or the other.

Most practicing computer scientists spend time taking advantage of the
known results about mathematical objects and structures, and rarely (if ever)
have to construct a water-tight proof about them. For the more
theoretically-minded student, however, who enjoys probing the basis behind
the tools and speculating about additional properties that might exist,
devising proofs is an essential skill that can also be very rewarding.


\section{Exercises}

\begin{small}
\begin{enumerate}
\newcolumntype{Q}{>{\arraybackslash}m{.45\textwidth}}
\newcolumntype{A}{>{\arraybackslash}m{.5\textwidth}}
%\begin{longtable}{m{0.3\textwidth} || m{0.6\textwidth}}
\begin{longtable}{Q || A}
\hline
\vspace{-.2in}
\item At a swim meet, the competitors in the 100-m freestyle are Ben, Chad,
Grover, and Tim. These four swimmers make up our sample space $\Omega$ for
the winner of this heat. \\
\\
Is Chad $\in \Omega$?
&
Yes.
\\
\hline
\item Is Tim an outcome?
&
Yes.
\\
\hline
\item Is Ben an event?
&
No, since outcomes are \textit{elements} of the sample space, while events
are \textit{subsets} of the sample space.
\\
\hline
\item Is \{~Chad, Grover~\} an event?
&
Yes.
\\
\hline
\item Is \{~Ben~\} an event?
&
Yes.
\\
\hline
\item Suppose I told you that Pr(\{Ben\})=.1, Pr(\{Chad\})=.2, Pr(\{Grover\})=.3,
and Pr(\{Tim\})=.3. Would you believe me?
&
Better not. This is not a valid probability measure, since the sum of the
probabilities of all the outcomes, Pr($\Omega$), is not equal to 1. 
\\
\hline
\item Suppose I told you that Pr(\{Ben, Chad\})=.3, and Pr(\{Ben, Tim\})=.4, and
Pr(\{Grover\})=.4. Could you tell me the probability that Ben wins the
heat?
&
Yes. If Pr(\{Ben, Chad\})=.3 and Pr(\{Grover\})=.4, that leaves .3
probability left over for Tim. And if Pr(\{Ben, Tim\})=.4, this implies
that Pr(\{Ben\})=.1.
\\
\hline
\item And what's the probability that someone besides Chad wins?
&
Pr($\overline{\{\text{Chad}\}}$) = $1 - $Pr(\{Chad\}), so we just need to
figure out the probability that Chad wins, and take one minus that. Clearly
if Pr(\{Ben, Chad\})=.3 (as we were told), and Pr(\{Ben\})=.1 (as we
computed), then Pr(\{Chad\})=.2, and the probability of a non-Chad winner
is .8.
\\
\hline
\item Okay, so we have the probabilities of our four swimmers Ben, Chad,
Grover, and Tim each winning the heat at .1, .2, .4, and .3, respectively.\\
Now suppose Ben, Chad, and Grover are UMW athletes, Tim is from Marymount,
Ben and Tim are juniors, and Chad and Grover are sophomores. We'll
define $U$=\{Ben,Chad,Grover\}, $M$=\{Tim\}, $J$=\{Ben,Tim\}, and
$S$=\{Chad,Grover\}. \\
\\
What's Pr($U$)? \label{pru}
&
.7.
\\
\hline
\item What's Pr($J$)? \label{prj}
&
.4.
\\
\hline
\item What's Pr($\overline{U}$)? \label{prnotu}
&
.3. (1 - Pr($U$), of course.)
\\
\hline
\item What's Pr($J \cup S$)?
&
Exactly 1. All of the outcomes are represented in the two sets $J$ and $S$.
(Put another way, all competitors are juniors or seniors.)
\\
\hline
\item What's Pr($J \cap S$)?
&
Zero. Sets $J$ and $S$ have no elements in common, therefore their intersection
is a set with no outcomes, and the probability of a non-existent outcome
happening is 0. (Put another way, nobody is both a junior and a senior.)
\\
\hline
\item What's the probability of a UMW junior winning the heat?
&
This is Pr($U \cap J$), which is the probability that the winner is a junior
\textit{and} a UMW student. Since $U \cap J =$ \{ Ben \}, the answer is .1.
\\
\hline
\item What's the probability that the winner is from UMW or a junior (or both)?
&
This is Pr($U \cup J$), which is the probability that the winner is a junior
\textit{or} a UMW student (or both). This calls for computing Pr($U$) plus
Pr($J$), but don't forget to then subtract Pr($U \cap J$) so we don't
double-count Ben! The correct answer is .7 + .4 - .1, which is equal to
\textbf{1}. If this surprises you, look again at the data and realize that
\textit{every} swimmer is either a UMW student (Chad and Grover),
a junior (Tim), or both (Ben).
\\
\hline
\item What's Pr($J|U$)?
&
By the definition of conditional probability, $\textrm{Pr}(J|U) =
\frac{\textrm{Pr}(J \cap U)}{\textrm{Pr}(U)}$ or $\frac{.1}{.7} = \frac{1}{7}
\approx \textbf{.143}.$ This is quite a bit lower than the .4 we computed for Pr($J$) in
item \ref{prj}. So if you knew nothing about the winner other than the
swimmers' baseline probabilities, you'd estimate a 40\% chance of a junior
winning...but if you learned the winner was a UMW student, your estimate of a
junior winner would drop down to nearly 14\%.
\\
\hline
\item What's Pr($\overline{U}|J$)?
&
$\textrm{Pr}(\overline{U}|J) = \frac{\textrm{Pr}(\overline{U} \cap
J)}{\textrm{Pr}(J)}$ or $\frac{.3}{.4} = \frac{3}{4} = \textbf{.75}$, way higher than
the .3 from item \ref{prnotu}. Learning that the swimmer is a junior makes the
likelihood of a non-UMW winner leap sky high.\\
\\
\hline

\item Suppose 75\% of Twitter users vote, whereas only about half of people in
general vote. Now say that about one out of every three people are on Twitter.
If you see someone emerge from a voting booth, what's the probability
they have a Twitter account?

&
The relevant probabilities here are Pr(vote$|$Twitter) = .75, Pr(vote) = .5,
and Pr(Twitter) = $\frac{1}{3}$. By Bayes' Theorem, Pr(Twitter$|$vote) =
$\frac{\textrm{Pr(vote}|\textrm{Twitter)}\cdot\textrm{Pr(Twitter)}}{\textrm{Pr(vote)}} =
\frac{.75 \cdot .333}{.5} = \textbf{.5}.$ So although only a third of people tweet,
the chances are 50-50 that someone tweets once you see them coming out of a
voting booth.
\\
\hline
\end{longtable}
\end{enumerate}
\end{small}


\section{Exercises}

\begin{small}
\begin{enumerate}
\newcolumntype{Q}{>{\arraybackslash}m{.45\textwidth}}
\newcolumntype{A}{>{\arraybackslash}m{.5\textwidth}}
%\begin{longtable}{m{0.3\textwidth} || m{0.6\textwidth}}
\begin{longtable}{Q || A}
\hline
\vspace{-.2in}
\item Inside a dusty chest marked ``Halloween costumes'' in the family attic,
there are four different outfits (a wizard's cape, army fatigues, and two
others), five different headgears (a batman helmet, a headband, a tiara,
\textit{etc.}), and nine different accessories (a wand, a lightsaber, a pipe,
and many others). If a child were to choose a costume by selecting one outfit,
one headgear, and one accessory, how many costume choices would he/she have?
&
$4 \times 5 \times 9 = 180$.\\
\hline
\item What if the child were permitted to skip one or more of the items (for
instance, choosing a costume with an outfit and accessory, but no headgear)?
&
$5 \times 6 \times 10 = 300$, since now ``no choice at all'' is effectively
another choice for each of the categories.\footnote{Note, by the way, that
this approach does \textit{not} work for situations like the license plate
example on p.\pageref{license plates}. Namely, you can't say ``if a license
plate can have fewer than 7 characters, we can just add `no character at this
position' as one of the options for that position,'' and calculate $37^7 =$
94,931,877,133 possible plates. That number is too high. Why? (Hint: for some
of those choices, you can get the same license plate in more than one way. Hint
2: if we choose 'A' for the first license plate character, `no character' for
the second, followed by 'NTMAN' we get the license plate 'ANTMAN'. But what other
choices could we have made, that would also have resulted in 'ANTMAN'?)
\index{license plates}} Kind of amazing how much that increases the total!\\
\hline
\item Go back to when the child did have to choose something from each
category, but now say they can have \textit{any} number of accessories (so they
could have the wizard's cape, a batman helmet, plus a lightsaber, pipe, and
scepter). Now how many costumes are there?
&
This is $4 \times 5 \times 2^9$, or a whopping 10,240 for those of you keeping
score. The 9 changed to a $2^9$ because now for \textit{each} accessory, a
costume might include it, or exclude it. That's two independent choices for
each accessory.\\ \hline \item Okay, that's overkill. A kid only has two hands,
after all, so handling nine accessories would be a dextrous challenge. Let's
say instead that a child can choose \textit{up to three} accessories (but must
have at least one). Now how many costume choices are there?
&
Now it's $4 \times 5 \times (\binom{9}{1} + \binom{9}{2} + \binom{9}{3})$,
which is equal to $4 \times 5 \times (9 + 36 + 84)$, or 2,580 possible costumes.\\
\hline
\item When it's finally time to go trick-or-treating, we join up with our
next-door neighbors and split up the families into somewhat haphazard groups.
There are eleven total children, and six adults. Now let's say each group must
have between 3 and 5 members, and must have at least one adult (to stay safe)
and at least one kid (otherwise what's the point?) How many different groups
are possible?
&
\index{complement, total (of sets)} Ignoring the at-least-one-child-and-adult
constraint for the moment, the total number of groups would seem to be
$\binom{17}{3} + \binom{17}{4} + \binom{17}{5} = 680 + 2380 + 6188 = 9,248$
possible groups. But of course this is an overcount, since it includes groups
with no children and groups with no adults. We'll use the trick from
p.~\pageref{one minus trick} to subtract those out. How many size-3-to-5 groups
with no adults (all kids) are there? $\binom{11}{3} + \binom{11}{4} +
\binom{11}{5} = 957$. And how many size-3-to-5 groups with no kids (all
adults)? $\binom{6}{3} + \binom{6}{4} + \binom{6}{5} = 41$. Therefore, by the
p.~\pageref{one minus trick} trick, the total number of legal groups is $9248 -
957 - 41 = 8,250$. Final answer.
\\
\hline
\vspace{-.2in}
\item To encourage rivalry and gluttony, we're going to give a special
certificate to the child who collects the most candy at the end of the night.
And while we're at it, we'll give 2nd-place and 3rd-place certificates as well.
How many different ways could our 1st-2nd-3rd contest turn out?
&
This is a partial permutation: there are eleven possible winners, and ten
possible runners-up for each possible winner, and nine possible 3rd-placers for
each of those top-twos. The answer is therefore $11^{\underline{3}}$, or 990.
Wow! I wouldn't have guessed that high.\\
\hline
\vspace{-.2in}
\item Finally, what if we want \textit{every} kid to get a certificate with
their name and place-of-finish on it. How many possibilities? (Assume no ties.)
&
This is now a full-blown permutation: $11!$. It comes to 39,916,800 different
orders-of-finish, believe it or not. I told you: this counting stuff can
explode fast.\\
\hline
\end{longtable}
\end{enumerate}
\end{small}

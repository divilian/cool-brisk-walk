
\section{Exercises}

\begin{small}
\begin{enumerate}
\newcolumntype{Q}{>{\arraybackslash}m{.45\textwidth}}
\newcolumntype{A}{>{\arraybackslash}m{.5\textwidth}}
%\begin{longtable}{m{0.3\textwidth} || m{0.6\textwidth}}
\begin{longtable}{Q || A}
\hline
\vspace{-.3in}
\item If I told you that the decimal number (\textit{i.e.}, base-10 number)
2022 was equal to $13621_6$, would you call me a liar without even having to
think too hard?
&
Yes, you should. A number in base-6 can't have any digits in it other than 0
through 5, and the ``number'' I tried to give you had a 6 in it.\\
\hline
\vspace{-.2in}
\item If I told you that the decimal number 2022 was
equal to $1413_6$, would you call me a liar without even having to think too
hard?
&
Yes, you should. Think about it: in base 6, each digit's place value (except
the one's place) is worth \textit{less} than it is in base 10. Instead of a
ten's place, hundred's place, and thousand's place, we have a woosy six's
place, thirty-six's place, and two-hundred-and-sixteen's place. So there's no
way that a number whose base-6 digits are 1, 4, 1, and 3 would be as large as a
number whose base-\textsl{10} digits are 2, something, something, and
something. Put another way, if the base is smaller, the number itself has to
``look bigger'' to have a chance of evening that out.\\
\hline
\vspace{-.2in}
\item If I told you that the decimal number 2022 was
equal to 8FA$8_{16}$, would you call me a liar without even having to think too
hard?
&
Yes, you should, because of the mirror reflection of the above logic.
Every digit of a hexadecimal number (again, except the one's place) is worth
\textit{more} than it is in base 10. So a four-digit hex number beginning with
an 8 is going to be way bigger than a wimpy four-digit decimal number beginning
with 2.\\
\hline
\vspace{-.2in}
\item If I told you that the decimal number 2022 was
equal to 1231$0_{6}$, would you call me a liar without even having to think too
hard?
&
No, you shouldn't, because you do have to think hard for this one. As it
happens, I \textit{am} a liar (the true answer is 1321$0_6$), but there's no
easy way to know that at a glance.\\
\hline
\vspace{-.2in}
\item If I told you that the decimal number 2022 was
equal to 7E$6_{16}$, would you call me a liar without even having to think too
hard?
&
No, you shouldn't, because you do have to think hard for this one. (And in
fact, it's correct! Work it out.)\\
\hline
\vspace{-.2in}
\item If I told you that 98,243,917,215 mod 7 was equal to 1, would you call me
a liar without even having to think too hard?
&
No, you shouldn't. It turns out that the answer is 3, not 1, but how would you
know that without working hard for it?\\
\hline
\vspace{-.2in}
\item If I told you that 273,111,999,214 mod 6 was equal to 6, would you call me
a liar without even having to think too hard?
&
Yes, you should. Any number mod 6 will be in the range 0 through 5, never 6 or
above. (Think in terms of repeatedly taking out groups of six from the big
number. The mod is the number of stones you have left when there are no more
whole groups of six to take. If towards the end of this process there are six
stones left, that's not a remainder, because you can get another whole
group!)\\
\hline
\vspace{-.2in}
\item Are the numbers 18 and 25 equal?
&
Of course not. Don't waste my time.\\
\hline
\item Are the numbers 18 and 25 congruent mod 7?
&
Yes. If we take groups of 7 out of 18 stones, we'll get two such groups (a
total of 14 stones) and have 4 left over. And then, if we do that same with 25
stones, we'll get three such groups (a total of 21 stones) and again have 4
left over. So they're not congruent mod 7.\\
\hline
\item Are the numbers 18 and 25 congruent mod 6?
&
No. If we take groups of \textit{6} out of 18 stones, we'll get three such
groups with nothing left over. But if we start with 25 stones, we'll take out 4
such groups (for a total of 24 stones) and have one left over. So they're not
congruent mod 6.\\
\hline
\vspace{-.2in}
\item Are the numbers 617,418 and 617,424 equal?
&
Of course not. Don't waste my time.\\
\hline
\item Are the numbers 617,418 and 617,424 congruent mod 3?
&
Yes. The number 617,418 is exactly 6 less than 617,424. Let's say there are $k$
stones left over after removing groups of three from 617,418. ($k$ must be 0,
1, or 2, of course.) Now if we did the same remove-groups-of-three thing
starting with 617,424, we'll have two more groups of three than we did before,
but then also have exactly $k$ stones left over.\\
\hline
\item Are the numbers 617,418 and 617,424 congruent mod 2?
&
Yes. The number 617,418 is exactly 6 less than 617,424. If there are $k$ stones
left over after removing pairs of stones from 617,418, we'd get three
additional pairs if we instead started with 617,424, but then also have exactly
$k$ stones left over.\\
\hline
\item Are the numbers 617,418 and 617,424 congruent mod 5?
&
No. Five doesn't go evenly into six.\\
\hline
\item Are the numbers 617,418 and 617,424 congruent mod 6?
&
Yes. The number 617,418 is exactly 6 less than 617,424. If there are $k$ stones
left over after removing groups of six from 617,418, we'd get one
additional group if we instead started with 617,424, and then have exactly
$k$ stones left over.\\
\hline
\end{longtable}
\end{enumerate}
\end{small}

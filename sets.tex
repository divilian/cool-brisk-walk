
\chapter{Sets}
\label{chap:sets}

\index{Cantor, Georg}
The place from which we'll start our walk is a body of mathematics called
``set theory." Set theory has an amazing property: it's so simple and
applicable that almost all the rest of mathematics can be based on it! This
is all the more remarkable because set theory itself came along pretty late
in the game (as things go) --- it was singlehandedly invented by one
brilliant man, Georg Cantor, in the 1870's. That may seem like a long time
ago, but consider that by the time Cantor was born, mankind had already
accumulated an immense wealth of mathematical knowledge: everything from
geometry to algebra to calculus to prime numbers. Set theory was so elegant
and universal, though, that after it was invented, nearly everything in
math was redefined from the ground up to be couched in the language of
sets. It turns out that this simple tool is an amazingly powerful way to
reason about mathematical concepts of all flavors. Thus everything else in
this book stands on set theory as a foundation.

Cantor, by the way, went insane as he tried to extend set theory to fully
encompass the concept of infinity. Don't let that happen to you.



\section{The idea of a set}

A \textbf{set} \index{sets} is a selection of certain things out of a
(normally larger) group.  When we talk about a set, we're declaring that
certain specific items from that group are \textit{in} the set, and certain
items are \textit{not} in the set. There's no shades of gray: every element
is either in or out. \index{elements (of sets)}

For instance, maybe the overall group I'm considering is my family, which
consists of five people: Dad, Mom, Lizzy, T.J., and Johnny. We could
define one set --- call it $A$ --- that contains Dad and Lizzy, but not
the other three. Another set $B$ might have Lizzy, T.J., and Johnny in it,
but not the two parents. The set $C$ might have Dad and \textit{only}
Dad in it.  The set $D$ might have all five Davieses, \index{Davies
family} and the set $E$
might have nobody at all. \textit{Etc.} You can see that every set is just
a way of specifying which elements are in and which are out.

Normally a set will be based on some property of its members, rather than
just being some random assortment of elements. That's what makes it worth
thinking about.  For example, the set $P$ (for ``parents") might be ``all
the Davieses who are parents": this set would contain Dad and Mom, and
no one else. The set $F$ (for ``female") might be declared as the female
members, and contain Mom and Lizzy. The set $H$ (for ``humans") would
contain all five elements of the group. And so on.

As with most of math, it turns out to be useful to define symbols for these
concepts, because then we can talk about them more precisely and concisely.
We normally list the members of a set using curly braces, like this:
\[
A = \{~\text{Dad}, \text{Lizzy}~\}
\]
or
\[
B = \{~\text{Lizzy}, \text{T.J.}, \text{Johnny}~\}
\]
Note that it doesn't matter what order you list the members in. The set
$F$ of females contains Mom and Lizzy, but it's not like Mom is the
``first" female or anything. That doesn't even make any sense. There is no
``first." A set's members are all equally members. So $P$ is the same
whether we write it like this:
\[
P = \{~\text{Dad}, \text{Mom}~\}
\]
or this:
\[
P = \{~\text{Mom}, \text{Dad}~\}.
\]
Those are just two different ways of writing the same thing.

The set $E$ that had nobody in it can be written like this, of course:
\[
E = \{~\}
\]
but we sometimes use this special symbol instead:
\[
E = \varnothing.
\]
However you write it, this kind of set (one that has no elements) is
referred to as an \textbf{empty set}.\index{empty set}

The set $H$, above, contained \textit{all} the members of the group under
consideration. Sometimes we'll refer to ``the group under consideration" as
the ``domain of discourse." \index{domain of discourse ($\Omega$)} It too is a set, and we usually use the symbol
$\Omega$ to refer to it.\footnote{Some authors use the symbol $U$ for this,
and call it the ``universal set."\index{universal set}} So in this case,
\[
\Omega = \{~\text{Mom}, \text{Johnny}, \text{T.J.}, \text{Dad},
\text{Lizzy}~\}.
\]
Another symbol we'll use a lot is ``$\in$", which means ``is a member
of." \index{member (of set)} Since Lizzy is a female, we can write:
\[
\text{Lizzy} \in F
\]
to show that Lizzy is a member of the $F$ set. Conversely, we write:
\[
\text{T.J.} \notin F
\]
to show that T.J. is not. 

As an aside, I mentioned that every item is either in, or not in, a set:
there are no shades of gray. Interestingly, researchers have developed
another body of mathematics called (I kid you not) ``fuzzy set theory."
\index{sets, fuzzy} Fuzzy sets change this membership assumption: items can
indeed be ``partially in" a set. One could declare, for instance, that
Dad is ``10\% female," which means he's only 10\% in the $F$ set. That
might or might not make sense for gender, but you can imagine that if we
defined a set $T$ of ``the tall people," such a notion might be useful. At
any rate, this example illustrates a larger principle which is important to
understand: in math, things are the way they are simply because we've
decided it's useful to think of them that way. If we decide there's a
different useful way to think about them, we can define new assumptions and
proceed according to new rules. It doesn't make any sense to say ``sets are
(or aren't) \textit{really} fuzzy": because there is no ``really." All
mathematics proceeds from whatever mathematicians have decided is useful to
define, and any of it can be changed at any time if we see fit.

\section{Defining sets}

\index{extensional} 
\index{intensional}
There are two ways to define a set: \textbf{extensionally} and
\textbf{intensionally}\footnote{Spelling nit: ``inten\textbf{s}ionally" has
an `s' in it. ``Inten\textbf{t}ionally," meaning ``deliberately," is a
completely different word.}. I'm not saying there are two kinds of sets:
rather, there are simply two \textit{ways to specify} a set.

To define a set extensionally is to list its actual members. That's what we
did when we said $P = \{~\text{Dad}, \text{Mom}~\}$, above. In this
case, we're not giving any ``meaning" to the set; we're just mechanically
spelling out what's in it. The elements Dad and Mom are called the
\textit{extension} \index{extensional} of the set $P$.

The other way to specify a set is intensionally, which means to describe
its meaning. Another way to think of this is specifying a rule by which it
can be determined whether or not a given element is in the set. If I say
``Let $P$ be the set of all parents," I am defining $P$ intensionally. I
haven't explicitly said which specific elements of the set are in $P$. I've
just given the meaning of the set, from which you can figure out the
extension. We call ``parent-ness" the \textit{intension} \index{intensional}
of $P$.

Note that two sets with different intensions might nevertheless have
the same extension. Suppose $O$ is ``the set of all people over 25 years
old" and $R$ is ``the set of all people who wear wedding rings." If our
$\Omega$ is the Davies family, then $O$ and $R$ have the same extension
(namely, Mom and Dad). They have different intensions, though:
conceptually speaking, they're describing different things. One could
imagine a world in which older people don't all wear wedding rings, or one
in which some younger people do. Within the domain of discourse of the
Davies family, however, the extensions happen to coincide.

Fact: we say two sets are equal \textit{if they have the same extension.}
\index{equality (for sets)} This might seem unfair to intensionality, but
that's the way it is. So it is totally legit to write:
\[
O = R
\]
since by the definition of set equality, they are in fact equal. I thought
this was weird at first, but it's really no weirder than saying ``the
number of years the Civil War lasted = Brett Favre's jersey number when he
played for the Packers." The things on the left and right side of that
equals sign refer conceptually to two very different things, but that
doesn't stop them from both having the value 4, and thus being equal.

\index{curly brace notation} 
\index{set-builder notation}
By the way, we sometimes use the curly brace notation in combination with a
colon to define a set intensionally. Consider this:
\[
M = \{~k : \text{$k$ is between 1 and 20, and a multiple of 3}~\}.
\]
When you reach a colon, pronounce it as ``such that." So this says ``$M$ is
the set of all numbers $k$ such that $k$ is between 1 and 20, and a
multiple of 3." (There's nothing special about $k$, here; I could have
picked any letter.) This is an intensional definition, since we haven't listed
the specific numbers in the set, but rather given a rule for finding them.
Another way to specify this set would be to write
\[
M = \{~3,6,9,12,15,18~\}
\]
which is an extensional definition of the same set.

\index{intensional} 
\index{extensional} 
Interesting thought experiment: what happens if you enlarge the intension
of a set by adding conditions to it?  Answer: increasing the intension
\textit{decreases} the extension. For example, suppose $M$ is initially
defined as the set of all males (in the Davies family). Now suppose I
decide to add to that intension by making it the set of all \textit{adult}
males. By adding to the intension, I have now reduced the extension from 
\{~Dad, T.J., Johnny~\} to just \{~Dad~\}.  The reverse is true as well:
trimming down the intension by removing conditions effectively increases
the extension of the set. Changing ``all male persons" to just ``all
persons" includes Mom and Lizzy in the mix.


\section{Finite and infinite sets}

\index{sets, infinite} \index{sets, finite} Sets can have an infinite
number of members. That doesn't make sense for the Davies family example,
but for other things it does, of course, like:
\[
I = \{~k : \text{$k$ is a multiple of 3}~\}.
\]
\index{Cantor, Georg}

Obviously there are infinitely many multiples of 3, and so $I$ has an
unlimited number of members. Not surprisingly, we call $I$ an \textbf{infinite
set}. More surprisingly, it turns out that there are different \textit{sizes}
of infinite sets, and hence different \textit{kinds} of infinity. For
instance, even though there are infinitely many whole numbers, and also
infinitely many real (decimal) numbers, there are nevertheless \textit{more}
real numbers than whole numbers. This is the thing that drove Cantor insane,
so we won't discuss it more here. For now, just realize that every set is
either finite or infinite. \index{infinity}

You might think, by the way, that there's no way to define an infinite set
extensionally, since that would require infinite paper. This isn't true,
though, if we creatively use an ellipsis: \index{ellipsis}
\[
I = \{~3,6,9,12,15,\dots~\}
\]
This is an extensional definition of $I$, since we're explicitly listing
all the members. It could be argued, though, that it's really intensional,
since the interpretation of ``\dots" requires the reader to figure out the
rule and mentally apply it to all remaining numbers. Perhaps in reality we
are giving an intensional definition, cloaked in an extensional-looking
list of members. I'm on the fence here.


\section{Sets are not arrays}

If you've done some computer programming, you might see a resemblance
between sets and the collections of items often used in a program: arrays,
perhaps, or linked lists. \index{arrays} \index{linked lists} To be sure,
there are some similarities. But there are also some very important
differences, which must not be overlooked:

\begin{itemize}

\item \textbf{No order.} \index{order (in sets)} As previously mentioned,
there is no order to the members of a set. ``\{Dad, Mom\}" is the same
set as ``\{Mom, Dad\}". In a computer program, of course, most arrays
or lists have first, second, and last elements, and an index number
assigned to each.

\item \textbf{No duplicates.} \index{duplicates (in sets)} Suppose $M$ is
the set of all males. What would it possibly mean to say $M$ = \{T.J.,
T.J., Johnny\}?  Would that mean that ``T.J. is twice the man that Johnny
is"? This is obviously nonsensical. The set $M$ is based on a property:
maleness. Each element of $\Omega$ is either male, or it isn't. It can't be
``male three times." Again, in an array or linked list, you could certainly
have more than one copy of the same item in different positions.

\item \textbf{Infinite sets.} 'Nuff said. I've never seen an array with
infinitely many elements, and neither will you. \index{sets, infinite}

\item \textbf{Untyped.} \index{typed} \index{untyped} \index{heterogeneous}
Most of the time, an array or other collection in a computer program contains
elements of only a single \textit{type}: it's an array of integers, or a
linked list of \texttt{Customer} objects, for example. This is important
because the program often needs to treat all elements in the collection the
same way. Perhaps it needs to loop over the array to add up all the numbers,
or iterate through a customer list and search for customers who have not
placed an order in the last six months. The program would run into problems if
it tried to add a string of text to its cumulative total, or encountered a
\texttt{Product} object in the middle of its list of \texttt{Customer}s. Sets,
though, can be \textbf{heterogeneous}, meaning they can contain different
kinds of things. The Davies family example had all human beings, but nothing
stops me from creating a set $X$ = \{~Jack Nicholson, Kim Kardashian,
Universal Studios, 5786, $\bigstar$~\}.

I don't press this point too hard for a couple of reasons. First, most
programming languages do allow heterogeneous collections of some sort, even if
they're not the most natural thing to express. In Java, you can define an
\texttt{ArrayList} as a non-generic so that it simply holds items of class
``\texttt{Object}." In C, you can have an array of \texttt{void *}'s ---
pointers to some unspecified type --- which allows your array to point to
different kinds of things. Unless it's a loosely-typed language,
\index{loosely-typed languages} though (like Perl or JavaScript), it sort of
feels like you're bending over backwards to do this. The other reason I make
this distinction lightly is that when we're dealing with sets, we often
\textit{do} find it useful to deal with things of only one type, and so our
$\Omega$ ends up being \textbf{homogeneous} \index{homogeneous} anyway.

\end{itemize}

Perhaps the biggest thing to remember here is that a set is a purely
abstract concept, whereas an array is a concrete, tangible, explicit list.
When we talk about sets, we're reasoning in general about large conceptual
things, whereas when we deal with arrays, we're normally iterating through
them for some specific purpose. You can't iterate through a set very easily
because (1) there's no order to the members, and (2) there might well be
infinitely many of them anyway.


\section{Sets are not ordered pairs (or tuples)}

You'll remember from high school algebra the notion of an \textbf{ordered
pair} $(x,y)$. \index{ordered pairs} We dealt with those when we wanted to
specify a point to plot on a graph: the first coordinate gave the distance
from the origin on the x-axis, and the second coordinate on the y-axis.
Clearly an ordered pair is not a set, because as the name implies it is
ordered: $(3,-4) \neq (-4,3)$. For this reason, we'll be very careful to
use curly braces to denote sets, and parentheses to denote ordered pairs.

By the way, although the word ``coordinate" is often used to describe the
elements of an ordered pair, that's really a geometry-centric word that
implies a visual plot of some kind. Normally we won't be plotting elements
like that, but we will still have use to deal with ordered pairs. I'll just
call the constituent parts ``elements" to make it more general.
\index{coordinates} \index{elements (of sets)}

Three-dimensional points need \textbf{ordered triple}s $(x,y,z)$,
\index{ordered triples} and it doesn't take a rocket scientist to deduce
that we could extend this to any number of elements. The question is what
to call them, and you \textit{do} sort of sound like a rocket scientist (or
other generic nerd) when you say \textbf{tuple}. \index{tuples} (Some
people rhyme this word with ``Drupal," and others with ``couple," by the
way, and there seems to be no consensus). If you have an ordered-pair-type
thing with 5 elements, therefore, it's a 5-tuple (or a quintuple). If it
has 117 elements, it's a 117-tuple, and there's really nothing else to call
it. The general term (if we don't know or want to specify how many
elements) is \textbf{n-tuple}. \index{n-tuples} In any case, it's an
ordered sequence of elements that may contain duplicates, so it's very
different than a set.


\section{Sets of sets}

Sets are heterogeneous \index{heterogeneous} --- a single set can contain four
universities, seven integers, and an ahi tuna --- and so it might occur to you
that they can contain other \textit{sets} as well. This is indeed true, but
let me issue a stern warning: you can get in deep water very quickly when you
start thinking about ``sets of sets." \index{sets of sets} In 1901, in fact,
the philosopher Bertrand Russell pointed out that this idea can lead to
unresolvable contradictions unless you put some constraints on it. What became
known as ``Russell's Paradox" \index{Russell's paradox} famously goes as
follows: consider the set $R$ of all sets that do \textit{not} have themselves
as members\footnote{For instance, the set $Z$ of all zebras is a member of
$R$, since $Z$ itself is a set (not a zebra) and so $Z \notin Z$. The set $S$,
on the other hand, defined as ``the set of all sets mentioned in this book,"
is \textit{not} a member of $R$, since $S$ contains itself as a member.}. Now
is $R$ a member of itself, or isn't it? Either way you answer turns out to be
wrong (try it!) which means that this whole setup must be flawed at some
level.

The good news is that as long as you don't deal with this kind of
self-referential loop (``containing yourself as a member") then it's pretty
safe to try at home. Consider this set:
\[
V = \{~3, 5, \{~5, 4~\}, 2~\}.
\]
This set has \textit{four} (not five) members. Three of $V$'s members are
integers: 2, 3, and 5. The other one is a set (with no name given). That
other set, by the way, has two members of its own: 4 and 5. If you were
asked, ``is $4 \in V$"? the answer would be \textit{no}.

As a corollary to this, there's a difference between \index{empty set}
\[
\varnothing
\]
and
\[
\{~\varnothing~\}.
\]

The former is a set with no elements. The latter is a set with \textit{one}
element: and that element just happens to be a set with nothing in it.


\section{Cardinality}

\index{cardinality (of sets)}

When we talk about the number of elements in a set, we use the word
\textbf{cardinality}. You'd think we could just call it the ``size" of the
set, but mathematicians sometimes like words that sound cool. The
cardinality of $M$ (the set of males, where the Davies family is the domain
of discourse) is 3, because there are three elements in it. The cardinality
of the empty set \index{empty set} $\varnothing$ is 0. The cardinality of
the set of all integers is $\infty$. \index{infinity} Simple as that.

The notation we use for cardinality is vertical bars, like with absolute
value. So we write: $|M| = 3$.

To restate the example immediately above, $|\varnothing| = 0$, but
$|\{\varnothing\}| = 1$.


\section{Some special sets}

In addition to the empty set, there are symbols for some other common sets,
including:

\begin{itemize}
\item $\mathbb{Z}$ --- \index{integers ($\mathbb{Z}$)} the integers (positive, negative, and zero)
\item $\mathbb{N}$ --- \index{natural numbers ($\mathbb{N}$)} the natural numbers (positive integers and zero)
\item $\mathbb{Q}$ --- \index{rational numbers ($\mathbb{Q}$)} the rational numbers (all numbers that can be
expressed as an integer divided by another integer)
\item $\mathbb{R}$ --- \index{real numbers ($\mathbb{R}$)} the real numbers
(all numbers that aren't imaginary, \index{imaginary numbers}
even decimal numbers that aren't rational) 
\end{itemize}

\index{Cantor, Georg}
The cardinality of all these sets is infinity, \index{infinity} although as I alluded to
previously, $|\mathbb{R}|$ is in some sense ``greater than" $|\mathbb{N}|$.
For the curious, we say that $\mathbb{N}$ is a \textbf{countably infinite}\index{infinite, countably}\index{infinite, uncountably} set, whereas $|\mathbb{R}|$ is \textbf{uncountably infinite}. Speaking very
loosely, this can be thought of this way: if we start counting up all the
natural numbers 0, 1, 2, 3, 4, \dots, we will never get to the end of them.
But \textit{at least we can start counting}. With the real numbers, we
can't even get off the ground. Where do you begin? Starting with 0 is fine,
but then what's the ``next" real number? Choosing anything for your second
number inevitably skips a lot in between. Once you've digested this, I'll
spring another shocking truth on you: $|\mathbb{Q}|$ is actually
\textit{equal} to
$|\mathbb{N}|$, not greater than it as $|\mathbb{R}|$ is. Cantor came up
with an ingenious numbering scheme whereby all the rational numbers ---
including 3, $-9$, $\frac{4}{17}$, and $-\frac{1517}{29}$ --- can be listed
off regularly, in order, just like the integers can. And so
$|\mathbb{Q}|=|\mathbb{N}|\neq|\mathbb{R}|$. This kind of stuff can
blow your mind.


\section{Combining sets}

Okay, so we have sets. Now what can we do with them? When you first learn
about numbers back before kindergarten, the next thing you learn is how to
combine numbers using various operations to produce other numbers. These
include $+, -, \times, \div,$ exponents, roots, \textit{etc.} Sets, too,
have operations that are useful for combining to make other sets. These
include:

\begin{itemize}

\index{set operators}
\index{or (logical operator)}
\item \textbf{Union} ($\cup$). \index{union (of sets)} The union of two
sets is a set that includes the elements that \textit{either (or both)} of
them have as members.  For instance, if $A$ = \{~Dad, Lizzy~\}, and $B$ =
\{~Lizzy, T.J., Johnny \}, then $A \cup B$ = \{~Dad, Lizzy, T.J.,
Johnny~\}. Note that an element is in the union if it is in $A$ \textit{or}
$B$. For this reason, there is a strong relationship between the union
operator of sets and the ``or" ($\vee$) operator of boolean logic that
we'll see later.

\index{and (logical operator)}
\item \textbf{Intersection} ($\cap$). \index{intersection (of sets)} The
intersection of two sets is a set that includes the elements that
\textit{both} of them have as members. In the above example, $A \cap B$ =
\{~Lizzy~\}. There is a strong connection between intersection and the
``and" ($\wedge$) boolean logic operator.

\item \textbf{(Partial) complement} ($-$). \index{complement, partial (of
sets)} Looks like subtraction, but significantly different. $A - B$
contains \textit{the elements from A that are not also in B}. So you start
with $A$, and then ``subtract off" the contents of $B$, if they occur. In
the above example, $A-B$ = \{~Dad~\}.  (Note that T.J.  and Johnny didn't
really enter in to the calculation.) Unlike $\cup$ and $\cap$, $-$ is not
\textbf{commutative}. \index{commutative} This means it's not symmetrical:
$A-B$ doesn't (normally) give the same answer as $B-A$. In this example,
$B-A$ is \{~T.J., Johnny~\}, whereas if you ever reverse the operands with
union or intersection, you'll always get the same result as before.

\item \label{complement} \textbf{(Total) complement} ($\overline{X}$).
\index{complement, total (of sets)} Same as the partial complement, above,
except that the implied first operand is $\Omega$. In other words, $A-B$ is
``all the things in $A$ that aren't in $B$," whereas $\overline{B}$ is
``all the things \textit{period} that aren't in $B$." Of course, ``all the
things period" means ``all the things that we're currently talking about."
The domain of discourse $\Omega$ \index{domain of discourse ($\Omega$)} is
very important here. If we're talking about the Davies family,
\index{Davies family} we would say that $\overline{M}$ = \{~Mom,
Lizzy~\}, because those are all the Davieses who aren't male. If, on the
other hand, $\Omega$ is ``the grand set of absolutely everything," then not
only is Mom a member of $\overline{M}$, but so is the number 12, the
French Revolution, and my nightmare last Tuesday about a rabid platypus.

\item \textbf{Cartesian product} ($\times$). \index{Cartesian product (of
sets)} Looks like multiplication, but \textit{very} different. When
you take the Cartesian product of two sets $A$ and $B$, you don't even get
the elements from the sets in the result. Instead, you get \textit{ordered
pairs} \index{ordered pairs} of elements. These ordered pairs represent
each combination of an element from A and an element from B. For instance,
suppose $A$ = \{~Bob, Dave~\} and $B$ = \{~Jenny, Gabrielle, and
Tiffany~\}. Then:

\begin{center}
$A \times B$ = \{
(Bob, Jenny), (Bob, Gabrielle), (Bob, Tiffany),\\
(Dave, Jenny), (Dave, Gabrielle), (Dave, Tiffany)~\}.
\end{center}

Study that list. The first thing to realize is that it consists of neither
guys nor girls, but of ordered pairs. (Clearly, for example, Jenny $\notin
A\times B$.) Every guy appears exactly once with every girl, and the guy is
always the first element of the ordered pair.  Since we have two guys and
three girls, there are six elements in the result, which is an easy way to
remember the $\times$ sign that represents Cartesian product. (Do not,
however, make the common mistake of thinking that $A \times B$ \textit{is}
6. $A \times B$ is a set, not a number. The cardinality of the set, of
course, is 6, so it's appropriate to write $|A \times B| = 6$.)

\end{itemize}

\subsection{Laws of combining sets}

There are a bunch of handy facts that arise when combining sets using the
above operators. The important thing is that these are all easily seen just
by thinking about them for a moment. Put another way, \textit{these aren't
facts to memorize; they're facts to look at and see for yourself.} They're
just a few natural consequences of the way we've defined sets and
operations, and there are many others.

\begin{itemize}

\item \textbf{Union and intersection are commutative.} \index{commutative}
As noted above, it's easy to see that $A \cup B$ will always give the same
result as $B \cup A$.  Same goes for $\cap$. (Not true for $-$, though.)

\item \textbf{Union and intersection are associative.} \index{associative}
``Associative" means that if you have an operator repeated several times,
left to right, it doesn't matter which order you evaluate them in. $(A \cup
B) \cup C$ will give the same result as $A \cup (B \cup C)$. This means we
can freely write expressions like ``$X \cup Y \cup Z$" and no one can
accuse us of being ambiguous. This is also true if you have three (or more)
intersections in a row. Be careful, though: associativity does \textit{not}
hold if you have unions and intersections mixed together. If I write $A
\cup B \cap C$ it matters very much whether I do the union first or the
intersection first. This is just how it works with numbers: $4 + 3 \times
2$ gives either 10 or 14 depending on the order of operations. In algebra,
we learned that $\times$ has precedence over +, and you'll always do that
one first in the absence of parentheses. We could establish a similar order
for set operations, but we won't: we'll always make it explicit with
parens.

\item \textbf{Union and intersection are distributive.}
\index{distributive}
\label{distributivelaw} You'll recall from basic algebra that $a \cdot
(b+c) = ab + ac$. Similarly with sets,
\[
X \cap (Y \cup Z) = (X \cap Y) \cup (X \cap Z).
\]
It's important to work this out for yourself rather than just memorize it
as a rule. Why does it work? Well, take a concrete example. Suppose $X$ is
the set of all female students, $Y$ is the set of all computer science
majors, and $Z$ is the set of all math majors. (Some students, of course,
double-major in both.) The left-hand side of the equals sign says ``first
take all the math and computer science majors and put them in a group.
Then, intersect that group with the women to extract only the female
students." The result is ``women who are either computer science majors or
math majors (or both)." Now look at the right-hand side. The first pair of
parentheses encloses only female computer science majors. The right pair
encloses female math majors. Then we take the union of the two, to get a
group which contains only females, and specifically only the females who
are computer science majors or math majors (or both). Clearly, the two
sides of the equals sign have the same extension.

The distributive property in basic algebra doesn't work if you flip the
times and plus signs (normally $a+b\cdot c \neq (a+b)\cdot (a+c)$), but
remarkably it does here:
\[
X \cup (Y \cap Z) = (X \cup Y) \cap (X \cup Z).
\]
Using the same definitions of $X$, $Y$, and $Z$, work out the meaning of
this one and convince yourself it's always true.

\item \textbf{Identity laws.} \index{identity laws (of sets)} Simplest thing you've learned all day: $X
\cup \varnothing = X$ and $X \cap \Omega = X$. You don't change $X$ by
adding nothing to it, or taking nothing away from it.

\item \textbf{Domination laws.} \index{domination laws (of sets)} The flip side of the above is that $X \cup
\Omega = \Omega$ and $X \cap \varnothing = \varnothing$. If you take $X$,
and then add everything and the kitchen sink to it, you get everything and
the kitchen sink. And if you restrict $X$ to having nothing, it of course
has nothing.

\item \textbf{Complement laws.} $X \cup \overline{X}  = \Omega$.
\index{complement laws (of sets)} This is another
way of saying ``everything (in the domain of discourse) \index{domain of
discourse ($\Omega$)} is either in, or
not in, a set." So if I take $X$, and then I take everything \textit{not}
in $X$, and smoosh the two together, I get everything. In a similar vein,
$X \cap \overline{X} = \varnothing$, \index{empty set} because there can't be any element that's
both in $X$ and not in $X$: that would be a contradiction. Interestingly,
the first of these two laws has become controversial in modern philosophy.
It's called ``the law of the excluded middle," \index{law of the excluded
middle} and is explicitly repudiated in many modern logic systems.

\item \textbf{De Morgan's laws.} \label{demorganslaws} \index{De Morgan's
laws} Now these are worth
memorizing, if only because (1) they're incredibly important, and (2) they
may not slip right off the tongue the way the previous properties do. The
first one can be stated this way:

\[
\overline{X \cup Y} = \overline{X} \cap \overline{Y}.
\]

Again, it's best understood with a specific example. Let's say you're
renting a house, and want to make sure you don't have any surly characters
under the roof. Let $X$ be the set of all known thieves. Let $Y$ be the set
of all known murderers. Now as a landlord, you don't want any thieves or
murderers renting your property. So who are you willing to rent to? Answer:
if $\Omega$ is the set of all people, you are willing to rent to $\overline{X
\cup Y}$.

Why that? Because if you take $X \cup Y$, that gives you all the
undesirables: people who are either murderers or thieves (or both). You
don't want to rent to any of them. In fact, you want to rent to the
\textit{complement} of that set; namely, ``anybody else." Putting an
overbar on that expression gives you all the non-thieves and non-murderers.

Very well. But now look at the right hand side of the equation. $\overline{X}$
gives you the non-thieves. $\overline{Y}$ gives you the non-murderers. Now in
order to get acceptable people, you want to rent only to someone who's
\textit{in both groups.} Put another way, they have to be both a non-thief
and a non-murderer in order for you to rent to them. Therefore, they must
be in the intersection of the non-thief group and the non-murderer group.
Therefore, the two sides of this equation are the same.

The other form of De Morgan's law is stated by flipping the intersections
and unions: \index{De Morgan's laws}
 
\[
\overline{X \cap Y} = \overline{X} \cup \overline{Y}.
\]

Work this one out for yourself using a similar example, and convince
yourself it's always true.

Augustus De Morgan, by the way, was a brilliant 19$^\text{th}$ century
mathematician with a wide range of interests. His name will come up again
when we study logic and mathematical induction.

\end{itemize}


\section{Subsets}

We learned that the ``$\in$" symbol is used to indicate set membership: the
element on the left is a member of the set on the right. A related but
distinct notion is the idea of a \textbf{subset}. \index{subsets} When we
say $X \subseteq Y$ (pronounced ``$X$ is a subset of $Y$"), it means that
\textit{every member of $X$ is also a member of $Y$.} The reverse is not
necessarily true, of course, otherwise ``$\subseteq$" would just mean
``$=$". So if $A$ = \{~Dad, Lizzy~\} and $K$ = \{~Dad, Mom, Lizzy~\},
then we can say $A \subseteq K$.

Be careful about the distinction between ``$\in$" and ``$\subseteq$", which
are often confused. With $\in$, the thing on the left is an
\textit{element}, \index{elements (of sets)} whereas with $\subseteq$, the thing on the left is a set.
This is further complicated by the fact that the element on the left-hand
side of $\in$ might well \textit{be} a set.

Let's give some examples. Suppose that $Q$ is the set \{~4, \{~9, 4~\}, 2
\}. $Q$ has three elements here, one of which is itself a set. Now suppose
that we let $P$ be the set \{~4, 9~\}. Question: is $P \in Q$? The answer
is yes: the set \{~4, 9~\} (which is the same as the set \{~9, 4~\}, just
written a different way) is in fact an element of the set $Q$. Next
question: is $P \subseteq Q$? The answer is no, $P \not\subseteq Q$. If $P$
were a subset of $Q$, that would imply that every member of $P$ (there are
two of them: 9 and 4) is also an element of $Q$, whereas in fact, only 4
is a member of $Q$, not 9. Last question: if $R$ is defined to be \{~2, 4
\}, is $R \subseteq Q$? The answer is yes, since both 2 and 4 are also
members of $Q$.

Notice that by the definition, every set is a subset of itself. Sometimes,
though, it's useful to talk about whether a set is really a \textit{sub}set
of another, and you don't want it to ``count" if the two sets are actually
equal. This is called a \textbf{proper subset}, \index{subsets, proper} and
the symbol for it is $\subset$. You can see the rationale for the choice of
symbol, because ``$\subseteq$" is kind of like ``$\leq$" for numbers, and
``$\subset$" is like ``$<$".

Every set is a subset (not necessarily a proper one) of $\Omega$, because
our domain of discourse \index{domain of discourse ($\Omega$)} by
definition contains everything that can come up in conversation. Somewhat
less obviously, the empty set \index{empty set} is a subset of every set.
It's weird to think that $\varnothing \subseteq Q$ when $Q$ has several
things in it, but the definition does hold. ``Every" member of
$\varnothing$ (there are none) is in fact also a member of $Q$.

One note about reading this notation that I found confusing at first.
Sometimes the expression ``$a \in X$" is pronounced ``$a$ is an element of
$X$," but other times it is read ``$a$, \textit{which is} an element of
$X$". This may seem like a subtle point, and I guess it is, but if you're
not ready for it it can be a extra stumbling block to understanding the
math (which is the last thing we need). Take this hypothetical (but quite
typical) excerpt from a mathematical proof:

\begin{center}
``Suppose $k \in \mathbb{N} < 10\dots$"
\end{center}

If you read this as ``Suppose $k$ \textit{is} a natural number \textit{is}
less than 10," it's ungrammatical. It really should be understood as
``Suppose $k$ (which is a natural number) is less than 10." This is
sometimes true of additional clauses as well. For instance, the phrase
``Suppose $k \in \mathbb{R} > 0$ is the x-coordinate of the first
point" should be read ``Suppose $k$, \textit{which is a real number
greater than zero}, is the x-coordinate of the first point."

I'll leave you with a statement about numbers worth pondering and
understanding:

\[
\varnothing \subset 
\mathbb{N} \subset
\mathbb{Z} \subset
\mathbb{Q} \subset
\mathbb{R} \subset
\Omega.
\]
\index{natural numbers ($\mathbb{N}$)}
\index{integers ($\mathbb{Z}$)}
\index{rational numbers ($\mathbb{Q}$)}
\index{real numbers ($\mathbb{R}$)}
\index{empty set}
\index{domain of discourse ($\Omega$)}


\section{Power sets}

\index{power sets}
\textbf{Power set} is a curious name for a simple concept. We talk about
the power set ``of" another set, which is \textit{the set of all subsets of
that other set.} Example: suppose $A$ = \{~Dad, Lizzy~\}. Then the power
set of $A$, which is written as ``$\mathbb{P}(A)$" is: \{~\{~Dad, Lizzy
\}, \{~Dad~\}, \{~Lizzy~\}, $\varnothing$~\}. Take a good look at all
those curly braces, and don't lose any. There are four elements to the
power set of $A$, each of which is one of the possible subsets. It might
seem strange to talk about ``\textit{all} of the possible subsets" --- when
I first learned this stuff, I remember thinking at first that there would
be no limit to the number of subsets you could make from a set. But of
course there is.  To create a subset, you can either include, or exclude,
each one of the original set's members. In $A$'s case, you can either (1)
include both Dad and Lizzy, or (2) include Dad but not Lizzy, or (3)
include Lizzy but not Dad, or (4) exclude both, in which case your subset
is $\varnothing$. Therefore, $\mathbb{P}(A)$ includes all four of those
subsets.

\index{cardinality (of sets)} 
Now what's the cardinality of $\mathbb{P}(X)$ for some
set $X$? That's an interesting question, and one well worth pondering. The
answer ripples through the heart of a lot of combinatorics
\index{combinatorics} and the binary number system, \index{binary numbers}
topics we'll cover later. And the answer is right at our fingertips, if we
just extrapolate from the previous example. To form a subset of $X$, we
have a choice to either \textit{in}clude, or else \textit{ex}clude, each of
its elements. So there's two choices for the first element\footnote{I know
there's really no ``first" element, but work with me here.}, and then
whether we choose to include or exclude that first element, there are two
choices for the second. Regardless of what we choose for those first two,
there are two choices for the third, \textit{etc.} So if $|X|=2$ (recall
that this notation means ``$X$ has two elements" or ``$X$ has a cardinality
of 2"), then its power set has $2 \times 2$ members. If $|X|=3$, then its
power set has $2 \times 2 \times 2$ members.  In general:

\[
|\mathbb{P}(X)| = 2^{|X|}.
\]

As a limiting case (and a brain-bender) notice that if $X$ is the empty
set, \index{empty set} then $\mathbb{P}(X)$ has \textit{one} (not zero)
members, because there is in fact \textit{one} subset of the empty set:
namely, the empty set itself. So $|X|=0$, and $|\mathbb{P}(X)|=1$. And that
jives with the above formula.


\section{Partitions}
\label{partitions}
\index{partitions}

\index{collectively exhaustive} 
\index{mutually exclusive}

Finally, there's a special variation on the subset concept called a
\textbf{partition}. A partition is a group of subsets of another set that
together are both \textbf{collectively exhaustive} and \textbf{mutually
exclusive}. This means that every element of the original set is in
\textit{one and only one} of the sets in the partition. Formally, a
partition of $X$ is a group of sets $X_1, X_2, \dots, X_n$ such that:

\[
X_1 \cup X_2 \cup \cdots \cup X_n = X,
\]
\text{and} 
\[
X_i \cap X_j = \varnothing \quad \text{for all $i$, $j$}.
\]

So let's say we've got a group of subsets that are supposedly a partition
of $X$. The first line, above, says that if we combine the contents of all
of them, we get everything that's in $X$ (and nothing more). This is called
being collectively exhaustive. The second line says that no two of the sets
have anything in common: they are mutually exclusive.

As usual, an example is worth a thousand words. \index{Davies family}
Suppose the set $D$ is \{ Dad, Mom, Lizzy, T.J., Johnny.~\} A
partition is any way of dividing $D$ up into subsets that meet the above
conditions. One such partition is:

\begin{center}
\{~Lizzy, T.J.~\},
\{~Mom, Dad~\}, and
\{~Johnny~\}.
\end{center}

Another one is:

\begin{center}
\{~Lizzy~\},
\{~T.J.~\},
\{~Mom~\}, and
\{~Johnny, Dad~\}.
\end{center}

Yet another is:

\begin{center}
$\varnothing$,
$\varnothing$,
\{~Lizzy, T.J., Johnny, Mom, Dad~\}, and
$\varnothing$.
\end{center}

All of these are ways of dividing up the Davies family into groups so that
no one is in more than one group, and everyone is in some group. The
following is \textit{not} a partition:

\begin{center}
\{~Mom, Lizzy, T.J.~\}, and
\{~Dad~\}
\end{center}

because it leaves out Johnny. This, too, is \textit{not} a partition:

\begin{center}
\{~Dad~\}, 
\{~Mom, T.J.~\}, and
\{~Johnny, Lizzy, Dad~\}
\end{center}

because Dad appears in two of the subsets.

By the way, realize that every set ($S$) together with its (total)
complement ($\overline{S}$) forms a partition of the entire domain of
discourse $\Omega$. \index{domain of discourse ($\Omega$)} This is because
every element either is, or is not, in any given set. The set of males and
non-males are a partition of $\Omega$ because everything is either a male
or a non-male, and never both (inanimate objects and other nouns are
non-males, just as women are). The set of prime numbers and the set of
everything-except-prime-numbers are a partition. The set of underdone
cheeseburgers and the set of everything-except-underdone-cheeseburgers form
a partition of $\Omega$.  By pure logic, this is true no matter what the
set is.

You might wonder why partitions are an important concept. The answer is
that they come up quite a bit, and when they do, we can make some important
simplifications. Take $S$, the set of all students at UMW. We can partition
it in several different ways. If we divide $S$ into the set of freshmen,
sophomores, juniors, and seniors, we have a partition: every student is one of
those grade levels, and no student is more than one.\footnote{Apologies to
fifth-year (or sixth-year, or...) ``super seniors.''} If we group them into
in-state and out-of-state students, we again have a partition. And if we divide
them into those who live on-campus and those who live off, we again have a
partition.

Note that dividing $S$ into computer science majors and English majors does
\textit{not} give us a partition. For one thing, not everyone is majoring in
one of those two subjects. For another, some students might be double-majoring
in both. Hence this group of subsets is neither mutually exclusive nor
collectively exhaustive. It's interesting to think about gender and partitions:
when I grew up, I was taught that males and females were a partition of the
human race. But now I've come to realize that there are non-binary persons who
do not identify with either of those genders, and so it's not a partition after
all.

\index{cardinality (of sets)}

Question: is the number of students $|S|$  equal to the number of off-campus
students plus the number of on-campus students? Obviously yes. But why? The
answer: because the off-campus and on-campus students form a partition. If we
added up the number of freshmen, sophomores, juniors, and seniors, we would
also get $|S|$. But adding up the number of computer science majors and English
majors would almost certainly \textit{not} be equal to $|S|$, because some
students would be double-counted and others counted not at all. This is an
example of the kind of beautiful simplicity that partitions provide.

